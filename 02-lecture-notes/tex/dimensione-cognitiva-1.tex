\documentclass[aspectratio=169,12pt]{beamer}
%
%=====================================================================
%
% Tema e colori
\usetheme{Madrid}
\usecolortheme{beaver}
\setbeamertemplate{navigation symbols}{}
%
%=====================================================================
%
% Pacchetti
\usepackage[italian]{babel}
\usepackage[utf8]{inputenc}
\usepackage{graphicx}
\usepackage{tikz}
\usepackage{booktabs}
\usepackage{enumitem}
%
%=====================================================================
%
\title{Dimensione Cognitiva \\ 1. Introduzione all'Intelligenza Artificiale}
\subtitle{Dalle origini ai paradigmi moderni}
\setbeamercovered{transparent} 
\author{Giovanni Della Lunga\\{\footnotesize giovanni.dellalunga@unibo.it}}
\institute{A lezione di Intelligenza Artificiale} 
\date{Siena - Giugno 2025} 
%
%=====================================================================
%
\begin{document}

% Slide titolo
\begin{frame}
    \titlepage
\end{frame}

% Indice
\begin{frame}{Indice}
    \tableofcontents
\end{frame}
%
%=====================================================================
%
\AtBeginSection[]
{
  %\begin{frame}<beamer>
  %\footnotesize	
  %\frametitle{Outline}
  %\begin{multicols}{2}
  %\tableofcontents[currentsection]
  %\end{multicols}	  
  %\normalsize
  %\end{frame}
  \begin{frame}
  \vfill
  \centering
  \begin{beamercolorbox}[sep=8pt,center,shadow=true,rounded=true]{title}  	 	 	\usebeamerfont{title}\insertsectionhead\par%
  \end{beamercolorbox}
  \vfill
  \end{frame}
}
\AtBeginSubsection{\frame{\subsectionpage}}
%__________________________________________________________________________
%
\section{Che cos'è l'Intelligenza Artificiale?}
%
%..................................................................
%
\begin{frame}{Definizione di Intelligenza Artificiale}
    \begin{block}{Definizione generale}
        L'\textbf{Intelligenza Artificiale} (IA o AI) è la disciplina che studia e sviluppa sistemi informatici capaci di eseguire compiti che normalmente richiederebbero intelligenza umana.
    \end{block}
    
    \vspace{0.5cm}
    
    \begin{itemize}
        \item Ragionamento logico
        \item Apprendimento dall'esperienza
        \item Riconoscimento di pattern
        \item Comprensione del linguaggio naturale
        \item Risoluzione di problemi complessi
        \item Percezione e interpretazione dell'ambiente
    \end{itemize}
\end{frame}
%
%..................................................................
%
\begin{frame}{L'AI nella vita quotidiana}
    \begin{columns}
        \begin{column}{0.5\textwidth}
            \textbf{Esempi concreti:}
            \begin{itemize}
                \item Assistenti vocali (Siri, Alexa)
                \item Raccomandazioni Netflix/Spotify
                \item Navigatori GPS
                \item Filtri antispam
                \item Traduttori automatici
                \item Fotocamere con riconoscimento facciale
                \item ChatGPT, Gemini, Claude ...
            \end{itemize}
        \end{column}
        \begin{column}{0.5\textwidth}
            \begin{center}
                \begin{tikzpicture}[scale=0.8]
                    \draw[fill=blue!20] (0,0) circle (2);
                    \node at (0,0.5) {\textbf{AI}};
                    \node at (0,0) {\textbf{nella}};
                    \node at (0,-0.5) {\textbf{vita}};
                    \node at (0,-1) {\textbf{quotidiana}};
                \end{tikzpicture}
            \end{center}
        \end{column}
    \end{columns}
\end{frame}
%__________________________________________________________________________
%
\section{Storia dell'Intelligenza Artificiale}
%
%..................................................................
%
\begin{frame}{Le origini: dai miti alla scienza}
    \begin{block}{Radici antiche}
        L'idea di creare esseri artificiali intelligenti ha radici antiche:
    \end{block}
    
    \begin{itemize}
        \item \textbf{Mitologia greca}: Talos (gigante di bronzo), Pandora
        \item \textbf{Medioevo}: automi meccanici nelle corti europee
        \item \textbf{1600-1700}: automi di Vaucanson, il "Turco meccanico"
        \item \textbf{1800}: Ada Lovelace e le prime idee di programmazione
    \end{itemize}
    
    \vspace{0.3cm}
    
    \begin{alertblock}{Punto di svolta}
        Il XX secolo porta le basi scientifiche: logica matematica, teoria della computazione, cibernetica
    \end{alertblock}
\end{frame}
%
%..................................................................
%
\begin{frame}{1950-1960: La nascita dell'AI moderna}
    \begin{block}{1950 - Alan Turing}
        \textbf{Test di Turing}: "Una macchina può pensare?"
    \end{block}
    
\begin{center}
\includegraphics[scale=.3]{../05-pictures/dimensione-cognitiva-1_pic_0.png} 
\end{center}

\end{frame}
%
%..................................................................
%
\begin{frame}{1950-1960: La nascita dell'AI moderna}
    
    \begin{block}{1956 - Conferenza di Dartmouth}
        \textbf{Nascita ufficiale} dell'AI come disciplina scientifica
        \begin{itemize}
            \item John McCarthy conia il termine "Artificial Intelligence"
            \item Partecipanti: Marvin Minsky, Herbert Simon, Allen Newell
            \item Obiettivo ambizioso: simulare ogni aspetto dell'intelligenza
        \end{itemize}
    \end{block}
    
    \vspace{0.3cm}
    
    \textbf{Primi successi:}
    \begin{itemize}
        \item Logic Theorist (1956) - dimostra teoremi matematici
        \item General Problem Solver (1957) - risolve problemi generici
    \end{itemize}
\end{frame}
%
%..................................................................
%
\begin{frame}{1950-1960: La nascita dell'AI moderna}
\begin{center}
\includegraphics[scale=.4]{../05-pictures/dimensione-cognitiva-1_pic_1.png} 
\end{center}
\end{frame}
%
%..................................................................
%
\begin{frame}{1950-1960: La nascita dell'AI moderna}
\begin{center}
\includegraphics[scale=.4]{../05-pictures/dimensione-cognitiva-1_pic_2.png} 
\end{center}
\end{frame}
%
%..................................................................
%
\begin{frame}{Anni '60-'70: Ottimismo e prime difficoltà}
    \begin{columns}
        \begin{column}{0.6\textwidth}
            \textbf{Grandi aspettative:}
            \begin{itemize}
                \item Previsioni di AI completa entro 20 anni
                \item Investimenti governativi massicci
                \item Sviluppo dei primi linguaggi AI (LISP)
            \end{itemize}
            
            \vspace{0.3cm}
            
            \textbf{Prime difficoltà:}
            \begin{itemize}
                \item Problemi più complessi del previsto
                \item Limitazioni computazionali
                \item "Esplosione combinatoriale"
            \end{itemize}
        \end{column}
        \begin{column}{0.4\textwidth}
            \begin{center}
                \textbf{Timeline}\\
                \vspace{0.2cm}
                \begin{tikzpicture}[scale=0.7]
                    \draw[thick] (0,0) -- (0,4);
                    \node[left] at (0,4) {1970};
                    \node[left] at (0,3) {1965};
                    \node[left] at (0,2) {1960};
                    \node[left] at (0,1) {1958};
                    \node[left] at (0,0) {1956};
                    
                    \draw[red] (-0.1,0) -- (0.1,0);
                    \node[right,scale=0.7] at (0.1,0) {Dartmouth};
                    \draw[red] (-0.1,1) -- (0.1,1);
                    \node[right,scale=0.7] at (0.1,1) {Perceptron};
                    \draw[red] (-0.1,3.5) -- (0.1,3.5);
                    \node[right,scale=0.7] at (0.1,3.5) {SHRDLU};
                \end{tikzpicture}
            \end{center}
        \end{column}
    \end{columns}
\end{frame}
%
%..................................................................
%
\begin{frame}{Anni '70-'80: Il primo "inverno dell'AI"}
    \begin{alertblock}{Crisi di fiducia (1974-1980)}
        \begin{itemize}
            \item Riduzione drastica dei finanziamenti
            \item Critiche ai limiti dei sistemi esistenti
            \item Report Lighthill (UK) molto critico
        \end{itemize}
    \end{alertblock}
    
    \vspace{0.3cm}
    
    \begin{block}{Rinascita con i Sistemi Esperti (1980-1987)}
        \textbf{Idea}: catturare la conoscenza degli esperti umani
        \begin{itemize}
            \item DENDRAL (analisi chimica)
            \item MYCIN (diagnosi mediche)
            \item R1/XCON (configurazione computer)
            \item Mercato da miliardi di dollari
        \end{itemize}
    \end{block}
\end{frame}
%
%..................................................................
%
\begin{frame}{Gli "Inverni" dell'AI}
\begin{center}
\includegraphics[scale=.75]{../05-pictures/dimensione-cognitiva-1_pic_3.png} 
\end{center}
\end{frame}
%
%..................................................................
%
\begin{frame}{Anni '90-2000: Approcci più realistici}
    \textbf{Cambiamento di paradigma:}
    \begin{itemize}
        \item Dai sistemi generali a quelli specializzati
        \item Focus su problemi specifici e misurabili
        \item Approcci statistici e probabilistici
    \end{itemize}
    
    \vspace{0.3cm}
    
    \textbf{Successi notevoli:}
    \begin{itemize}
        \item \textbf{1997}: Deep Blue batte Kasparov a scacchi
        \item Sviluppo del machine learning
        \item Nascita del web e dei big data
        \item Algoritmi di ricerca e raccomandazione
    \end{itemize}
    
    \vspace{0.3cm}
    
    \begin{block}{Fattori abilitanti}
        Maggiore potenza computazionale + Grandi quantità di dati + Algoritmi migliorati
    \end{block}
\end{frame}
%
%..................................................................
%
\begin{frame}{2000-oggi: L'era del deep learning}
    \textbf{Rivoluzioni recenti:}
    \begin{itemize}
        \item \textbf{2006}: Hinton e il deep learning
        \item \textbf{2012}: AlexNet rivoluziona la computer vision
        \item \textbf{2016}: AlphaGo batte Lee Sedol al Go
        \item \textbf{2017}: Transformer e l'NLP moderno
        \item \textbf{2022}: ChatGPT porta l'AI al grande pubblico
    \end{itemize}
    
    \vspace{0.3cm}
    
    \begin{alertblock}{Fattori chiave del successo attuale}
        \begin{itemize}
            \item GPU e calcolo parallelo massivo
            \item Internet e big data
            \item Algoritmi di apprendimento profondo
            \item Investimenti miliardari
        \end{itemize}
    \end{alertblock}
\end{frame}
%__________________________________________________________________________
%
\section{I due paradigmi dell'AI}
%
%..................................................................
%
\begin{frame}{AI Simbolica vs AI Subsimbolica}
    \begin{center}
        \textbf{L'AI non è solo Machine Learning!}
    \end{center}
    
    \vspace{0.5cm}
    
    \begin{columns}
        \begin{column}{0.5\textwidth}
            \begin{block}{\centering AI Simbolica}
                \centering
                \textbf{"Good Old-Fashioned AI"}\\
                \textbf{(GOFAI)}
            \end{block}
        \end{column}
        \begin{column}{0.5\textwidth}
            \begin{block}{\centering AI Subsimbolica}
                \centering
                \textbf{Machine Learning}\\
                \textbf{Deep Learning}
            \end{block}
        \end{column}
    \end{columns}
    
    \vspace{0.5cm}
    
    \begin{center}
        \begin{tikzpicture}
            \draw[thick, <->] (0,0) -- (8,0);
            \node[below] at (2,0) {Logica};
            \node[below] at (6,0) {Statistica};
            \node[above] at (2,0) {Regole esplicite};
            \node[above] at (6,0) {Apprendimento dai dati};
            \draw[red,thick] (2,-0.2) -- (2,0.2);
            \draw[blue,thick] (6,-0.2) -- (6,0.2);
        \end{tikzpicture}
    \end{center}
\end{frame}
%
%..................................................................
%
\begin{frame}{AI Simbolica: Il ragionamento logico}
    \begin{block}{Principi base}
        \begin{itemize}
            \item Rappresentazione esplicita della conoscenza
            \item Uso di simboli e regole logiche
            \item Ragionamento deduttivo
            \item Trasparenza e spiegabilità
        \end{itemize}
    \end{block}
    
    \textbf{Esempi concreti:}
    \begin{itemize}
        \item \textbf{Sistemi esperti}: MYCIN per diagnosi mediche
        \item \textbf{Pianificazione}: robot che pianifica movimenti
        \item \textbf{Dimostrazione di teoremi}: assistenti matematici
        \item \textbf{Elaborazione del linguaggio}: grammatiche formali
    \end{itemize}
\end{frame}
%
%..................................................................
%
\begin{frame}{AI Simbolica: Il ragionamento logico}
    \begin{block}{Principi base}
        \begin{itemize}
            \item Rappresentazione esplicita della conoscenza
            \item Uso di simboli e regole logiche
            \item Ragionamento deduttivo
            \item Trasparenza e spiegabilità
        \end{itemize}
    \end{block}
    
    \vspace{0.3cm}
    
    \begin{exampleblock}{Esempio: Regola medica}
        SE (febbre $\ge$  38°C) E (mal di gola) E (linfonodi gonfi)\\
        ALLORA probabilità streptococco = alta
    \end{exampleblock}
\end{frame}
%
%..................................................................
%
\begin{frame}{AI Simbolica: Vantaggi e Limiti}
    \begin{columns}
        \begin{column}{0.5\textwidth}
            \textbf{Vantaggi:}
            \begin{itemize}
                \item Trasparenza totale
                \item Facilità di debug
                \item Incorpora conoscenza esperta
                \item Ragionamento preciso
                \item Non servono grandi dataset
            \end{itemize}
        \end{column}
        \begin{column}{0.5\textwidth}
            \textbf{Limiti:}
            \begin{itemize}
                \item Difficile acquisire conoscenza
                \item Rigidità nelle regole
                \item Non gestisce incertezza
                \item Costosa da mantenere
                \item Non si adatta automaticamente
            \end{itemize}
        \end{column}
    \end{columns}
    
    \vspace{0.5cm}
    
    \begin{alertblock}{Il problema della "bottiglia della conoscenza"}
        Come trasferire la conoscenza dell'esperto umano nel sistema?
    \end{alertblock}
\end{frame}
%
%..................................................................
%
\begin{frame}{AI Subsimbolica: L'apprendimento dai dati}
    \begin{block}{Principi base}
        \begin{itemize}
            \item Apprendimento automatico da esempi
            \item Rappresentazioni distribuite (neuroni, pesi)
            \item Adattamento statistico
            \item Pattern recognition
        \end{itemize}
    \end{block}
    
    \textbf{Categorie principali:}
    \begin{itemize}
        \item \textbf{Machine Learning classico}: SVM, Decision Trees, k-means
        \item \textbf{Reti neurali artificiali}: Perceptron, MLP
        \item \textbf{Deep Learning}: CNN, RNN, Transformer
        \item \textbf{Apprendimento per rinforzo}: AlphaGo, giochi
    \end{itemize}
\end{frame}
%
%..................................................................
%
\begin{frame}{AI Subsimbolica: L'apprendimento dai dati}
    \begin{block}{Principi base}
        \begin{itemize}
            \item Apprendimento automatico da esempi
            \item Rappresentazioni distribuite (neuroni, pesi)
            \item Adattamento statistico
            \item Pattern recognition
        \end{itemize}
    \end{block}
    
    \vspace{0.3cm}
    
    \begin{exampleblock}{Esempio: Riconoscimento immagini}
        Sistema impara a riconoscere gatti analizzando migliaia di foto etichettate
    \end{exampleblock}
\end{frame}
%
%..................................................................
%
\begin{frame}{AI Subsimbolica: Vantaggi e Limiti}
    \begin{columns}
        \begin{column}{0.5\textwidth}
            \textbf{Vantaggi:}
            \begin{itemize}
                \item Apprende automaticamente
                \item Gestisce dati rumorosi
                \item Adattabile e flessibile
                \item Eccelle in pattern recognition
                \item Migliora con più dati
            \end{itemize}
        \end{column}
        \begin{column}{0.5\textwidth}
            \textbf{Limiti:}
            \begin{itemize}
                \item "Scatola nera"
                \item Servono molti dati
                \item Computazionalmente costoso
                \item Può overfittare
                \item Difficile da debuggare
            \end{itemize}
        \end{column}
    \end{columns}
    
    \vspace{0.5cm}
    
    \begin{alertblock}{Il problema della spiegabilità}
        Come sapere perché il sistema ha preso una certa decisione?
    \end{alertblock}
\end{frame}
%
%..................................................................
%
\begin{frame}{Confronto diretto: Simbolica vs Subsimbolica}
    \begin{center}
        \begin{tabular}{|l|c|c|}
            \hline
            \textbf{Caratteristica} & \textbf{Simbolica} & \textbf{Subsimbolica} \\
            \hline
            Trasparenza & \textcolor{green}{Alta} & \textcolor{red}{Bassa} \\
            \hline
            Dati richiesti & \textcolor{green}{Pochi} & \textcolor{red}{Molti} \\
            \hline
            Adattabilità & \textcolor{red}{Bassa} & \textcolor{green}{Alta} \\
            \hline
            Gestione rumore & \textcolor{red}{Difficile} & \textcolor{green}{Buona} \\
            \hline
            Conoscenza esperta & \textcolor{green}{Incorporabile} & \textcolor{red}{Difficile} \\
            \hline
            Scalabilità & \textcolor{red}{Limitata} & \textcolor{green}{Buona} \\
            \hline
            Costo computazionale & \textcolor{green}{Basso} & \textcolor{red}{Alto} \\
            \hline
        \end{tabular}
    \end{center}
    
    \vspace{0.5cm}
    
    \begin{block}{Conclusione}
        Nessun approccio è superiore in assoluto: dipende dal problema e dal contesto!
    \end{block}
\end{frame}
%__________________________________________________________________________
%
\section{Applicazioni e Futuro}
%
%..................................................................
%
\begin{frame}{Applicazioni moderne: Approcci ibridi}
    \textbf{La tendenza attuale}: combinare simbolica e subsimbolica
    
    \vspace{0.3cm}
    
    \begin{exampleblock}{Esempi di sistemi ibridi}
        \begin{itemize}
            \item \textbf{Diagnosi medica}: Deep learning per analisi immagini + regole cliniche
            \item \textbf{Veicoli autonomi}: CNN per percezione + pianificazione simbolica
            \item \textbf{Assistenti virtuali}: NLP neurale + knowledge base strutturati
            \item \textbf{Giochi}: reti neurali + ricerca ad albero (AlphaGo)
        \end{itemize}
    \end{exampleblock}
    
    \vspace{0.3cm}
    
    \textbf{Vantaggi dell'approccio ibrido:}
    \begin{itemize}
        \item Combina il meglio di entrambi i mondi
        \item Maggiore robustezza e affidabilità
        \item Spiegabilità dove necessaria
        \item Adattabilità dove richiesta
    \end{itemize}
\end{frame}
%
%..................................................................
%
\begin{frame}{Sfide attuali dell'AI}
    \begin{block}{Sfide tecniche}
        \begin{itemize}
            \item \textbf{Spiegabilità}: rendere l'AI più trasparente
            \item \textbf{Robustezza}: sistemi che funzionano in contesti diversi
            \item \textbf{Efficienza}: ridurre i costi computazionali
            \item \textbf{Generalizzazione}: AI che funziona oltre i dati di training
        \end{itemize}
    \end{block}
    
    \begin{block}{Sfide etiche e sociali}
        \begin{itemize}
            \item \textbf{Bias algoritmici}: pregiudizi nei dati e nei modelli
            \item \textbf{Privacy}: protezione dei dati personali
            \item \textbf{Lavoro}: impatto sull'occupazione
            \item \textbf{Sicurezza}: AI affidabile e controllabile
        \end{itemize}
    \end{block}
\end{frame}
%
%..................................................................
%
\begin{frame}{Implicazioni per l'educazione}
    \textbf{Come l'AI sta cambiando l'educazione:}
    
    \begin{itemize}
        \item \textbf{Personalizzazione}: sistemi di tutoring adattivi
        \item \textbf{Assistenza}: strumenti per correzione automatica
        \item \textbf{Accessibilità}: traduzione e trascrizione automatica
        \item \textbf{Nuove competenze}: necessità di alfabetizzazione digitale
    \end{itemize}
    
    \vspace{0.5cm}
    
    \begin{alertblock}{Ruolo cruciale dell'insegnante}
        L'AI non sostituisce l'insegnante, ma può potenziarne l'efficacia:
        \begin{itemize}
            \item Liberare tempo dalle attività ripetitive
            \item Focus su creatività, pensiero critico, relazioni umane
            \item Interpretazione e contestualizzazione delle informazioni
        \end{itemize}
    \end{alertblock}
\end{frame}
%
%..................................................................
%
\begin{frame}{Conclusioni}
    \begin{block}{Punti chiave da ricordare}
        \begin{itemize}
            \item L'AI ha radici antiche ma sviluppo recente accelerato
            \item Non è solo machine learning: esistono approcci diversi
            \item Simbolica e subsimbolica hanno vantaggi e limiti complementari
            \item Gli approcci ibridi rappresentano il futuro
            \item L'AI pone sfide tecniche, etiche e sociali importanti
        \end{itemize}
    \end{block}
    
    \vspace{0.5cm}
    
    \begin{center}
        \textbf{L'intelligenza artificiale è uno strumento potente\\
        che richiede comprensione, uso consapevole\\
        e considerazione etica}
    \end{center}
\end{frame}
%
%..................................................................
%
\begin{frame}{Bibliografia e Approfondimenti}
    \textbf{Testi introduttivi:}
    \begin{itemize}
        \item Russell, S., Norvig, P. "Artificial Intelligence: A Modern Approach"
        \item Flach, P. "Machine Learning: The Art and Science of Algorithms"
        \item Nilsson, N. "The Quest for Artificial Intelligence"
    \end{itemize}
    
    \vspace{0.3cm}
    
    \textbf{Risorse online:}
    \begin{itemize}
        \item Coursera: "Machine Learning" (Andrew Ng)
        \item MIT OpenCourseWare: "Introduction to AI"
        \item Elements of AI (University of Helsinki)
    \end{itemize}
    
    \vspace{0.3cm}
    
    \textbf{Per rimanere aggiornati:}
    \begin{itemize}
        \item IEEE Spectrum AI
        \item MIT Technology Review
        \item Nature Machine Intelligence
    \end{itemize}
\end{frame}


\end{document}