\documentclass[aspectratio=169]{beamer}
\usepackage[utf8]{inputenc}
\usepackage[italian]{babel}
\usepackage{amsmath}
\usepackage{graphicx}
\usepackage{tikz}
\usepackage{xcolor}
%
%=====================================================================
%
% Tema e colori
\usetheme{Madrid}
\usecolortheme{seahorse}
%
%=====================================================================
%
% Definizione colori personalizzati
\definecolor{aiblue}{RGB}{30,144,255}
\definecolor{mlgreen}{RGB}{46,139,87}
\definecolor{datacolor}{RGB}{255,140,0}
%
%=====================================================================
%
\title{Dimensione Cognitiva \\ 3. Reti Neurali}
\subtitle{Introduzione al Machine Learning}
\setbeamercovered{transparent} 
\author{Giovanni Della Lunga\\{\footnotesize giovanni.dellalunga@unibo.it}}
\institute{A lezione di Intelligenza Artificiale} 
\date{Siena - Giugno 2025} 
%
%=====================================================================
%
\begin{document}

% Slide titolo
\begin{frame}
    \titlepage
\end{frame}

% Indice
\begin{frame}{Indice}
    \tableofcontents
\end{frame}
%
%=====================================================================
%
\AtBeginSection[]
{
  %\begin{frame}<beamer>
  %\footnotesize	
  %\frametitle{Outline}
  %\begin{multicols}{2}
  %\tableofcontents[currentsection]
  %\end{multicols}	  
  %\normalsize
  %\end{frame}
  \begin{frame}
  \vfill
  \centering
  \begin{beamercolorbox}[sep=8pt,center,shadow=true,rounded=true]{title}  	 	 	\usebeamerfont{title}\insertsectionhead\par%
  \end{beamercolorbox}
  \vfill
  \end{frame}
}
\AtBeginSubsection{\frame{\subsectionpage}}
%__________________________________________________________________________
%
\section{I Limiti dei Confini Lineari}
%
%..................................................................
%
\begin{frame}{Dove Abbiamo Lasciato...}
\begin{block}{Recap: La Visione Geometrica}
\begin{itemize}
\item Ogni dato = punto nello spazio multidimensionale
\item Machine Learning = trovare confini ottimali
\item Classificazione = separazione geometrica
\end{itemize}
\end{block}

\pause

\begin{block}{La Formula Magica}
$$\text{Decisione} = \text{sign}(w_1x_1 + w_2x_2 + \ldots + w_nx_n + b)$$
\end{block}

\pause
\begin{alertblock}{Ma cosa succede quando...}
...i dati non sono linearmente separabili?
\end{alertblock}
\end{frame}
%
%..................................................................
%
\begin{frame}{Un Problema che Non Riusciamo a Risolvere}
\begin{columns}
\begin{column}{0.5\textwidth}
\textbf{Scenario: Riconoscimento Volti}
\begin{itemize}
\item $x_1$: Luminosità media
\item $x_2$: Contrasto dell'immagine
\end{itemize}
\vspace{.5cm}
\textbf{Il Problema:}\\
Non esiste una linea retta che separi ``volto'' da ``non volto''!
\end{column}
\begin{column}{0.5\textwidth}
\begin{center}
\includegraphics[scale=.6]{../05-pictures/dimensione-cognitiva-3_pic_0.png} 
\end{center}
\end{column}
\end{columns}

\end{frame}
%
%..................................................................
%
\begin{frame}{Un Problema che Non Riusciamo a Risolvere}

\textbf{La soluzione:}\\
Possiamo però trasformare i dati del problema!
\begin{center}
\includegraphics[scale=.6]{../05-pictures/dimensione-cognitiva-3_pic_1.png} 
\end{center}
\end{frame}
%
%..................................................................
%
\begin{frame}
\frametitle{Un problema apparentemente semplice}
\begin{center}
\includegraphics[scale=.45]{../05-pictures/dimensione-cognitiva-3_pic_2.png} 
\end{center}
\end{frame}
%
%..................................................................
%
\begin{frame}
\frametitle{Un problema apparentemente semplice}
\begin{center}
\includegraphics[scale=.45]{../05-pictures/dimensione-cognitiva-3_pic_3.png} 
\end{center}
\end{frame}
%
%..................................................................
%
\begin{frame}
\frametitle{Perché i Problemi Reali Sono Difficili}
\framesubtitle{Oltre i Confini Lineari}
\small
\begin{block}{Esempio: Distinguere Email Spam}
\textbf{Regola semplice (lineare):} "Se contiene più di 5 numeri → Spam"
\textcolor{red}{Problema:} Le email spam sono furbe!
\begin{itemize}
    \item Usano sinonimi: "Vincere" → "V1nc3r3"
    \item Cambiano strategia continuamente
    \item Imitano email legittime
\end{itemize}
\end{block}

\begin{block}{Esempio: Riconoscere Funghi Velenosi}
\textbf{Regola semplice:} "Se è rosso → Velenoso"
\textcolor{red}{Problema:} La natura è complessa!
\begin{itemize}
    \item Funghi rossi commestibili esistono
    \item Funghi bianchi possono essere mortali
    \item Serve combinare: colore + forma + habitat + stagione
\end{itemize}
\end{block}
\end{frame}
%__________________________________________________________________________
%
\section{Il Neurone Artificiale: Oltre la Linearità}
%
%..................................................................
%
\begin{frame}{L'Idea Rivoluzionaria}

\begin{block}{Ispirazione Biologica}
Il cervello umano processa informazioni attraverso miliardi di neuroni interconnessi
\end{block}

\begin{block}{Neurone Artificiale = Modello Lineare + Funzione Non-Lineare}
$$\text{output} = f\left(\sum_{i=1}^{n} w_i x_i + b\right)$$

dove $f$ è una \textbf{funzione di attivazione non-lineare}
\end{block}

\end{frame}
%
%..................................................................
%
\begin{frame}{L'Idea Rivoluzionaria}
\begin{center}
\includegraphics[scale=.45]{../05-pictures/dimensione-cognitiva-3_pic_4.png} 
\end{center}
\end{frame}
%
%..................................................................
%
\begin{frame}{L'Idea Rivoluzionaria}
\begin{center}
\includegraphics[scale=.45]{../05-pictures/dimensione-cognitiva-3_pic_5.png} 
\end{center}
\end{frame}
%
%..................................................................
%
\begin{frame}{L'Idea Rivoluzionaria}
\begin{center}
\includegraphics[scale=.45]{../05-pictures/dimensione-cognitiva-3_pic_6.png} 
\end{center}
\end{frame}
%
%..................................................................
%
\begin{frame}{L'Idea Rivoluzionaria}
\begin{center}
\includegraphics[scale=.45]{../05-pictures/dimensione-cognitiva-3_pic_7.png} 
\end{center}
\end{frame}
%
%..................................................................
%
\begin{frame}{L'Idea Rivoluzionaria}
\begin{center}
\includegraphics[scale=.45]{../05-pictures/dimensione-cognitiva-3_pic_8.png} 
\end{center}
\end{frame}
%
%..................................................................
%
\begin{frame}{L'Idea Rivoluzionaria}
\begin{center}
\includegraphics[scale=.35]{../05-pictures/dimensione-cognitiva-3_pic_9.png} 
\end{center}
\end{frame}
%
%..................................................................
%
\begin{frame}{Funzioni di Attivazione: Gli Interruttori Intelligenti}

\begin{columns}
\begin{column}{0.45\textwidth}
\textbf{Sigmoid:} $\sigma(z) = \frac{1}{1+e^{-z}}$
\begin{itemize}
\item Output tra 0 e 1
\item ``Interruttore morbido''
\item Interpretabile come probabilità
\end{itemize}

\vspace{0.5cm}
\textbf{ReLU:} $f(z) = \max(0,z)$
\begin{itemize}
\item Semplice ed efficace
\item ``Attiva o spegni''
\item Molto usata oggi
\end{itemize}
\end{column}

\begin{column}{0.45\textwidth}
\begin{center}
\begin{tikzpicture}[scale=0.7]
% Sigmoid
\draw[->] (-3,0) -- (3,0) node[right] {$z$};
\draw[->] (0,-0.5) -- (0,2.5) node[above] {$\sigma(z)$};
\draw[blue,thick,domain=-3:3,smooth] plot (\x,{2/(1+exp(-2*\x))});
\node[blue] at (1.5,2.2) {Sigmoid};

% Griglia per ReLU
\begin{scope}[yshift=-3.5cm]
\draw[->] (-3,0) -- (3,0) node[right] {$z$};
\draw[->] (0,-0.5) -- (0,2.5) node[above] {ReLU};
\draw[red,thick] (-3,0) -- (0,0);
\draw[red,thick] (0,0) -- (2.5,2.5);
\node[red] at (1.5,2.2) {ReLU};
\end{scope}
\end{tikzpicture}
\end{center}
\end{column}
\end{columns}

\end{frame}
%
%..................................................................
%
\begin{frame}
\frametitle{Il Neurone Come Decisore Intelligente}
\framesubtitle{Un'Analogia Familiare}

\begin{columns}
\begin{column}{0.6\textwidth}
\textbf{Immaginate un Preside che Decide le Sospensioni:}

\vspace{0.3cm}
\textbf{Input (Informazioni):}
\begin{itemize}
    \item Gravità del comportamento
    \item Storia precedente dello studente  
    \item Circostanze attenuanti
    \item Testimonianze di altri docenti
\end{itemize}

\vspace{0.3cm}
\textbf{Pesi (Importanza):}
\begin{itemize}
    \item Gravità: \textcolor{red}{peso alto}
    \item Storia: \textcolor{blue}{peso medio}
    \item Attenuanti: \textcolor{green}{peso basso}
\end{itemize}

\vspace{0.3cm}
\textbf{Decisione:} Sospensione SÌ/NO
\end{column}

\begin{column}{0.4\textwidth}
\begin{center}
\textbf{Formula del Preside:}

\vspace{0.3cm}
\fbox{\parbox{0.9\textwidth}{
\centering
Decisione = f(\\
\quad Gravità × 0.6 +\\
\quad Storia × 0.3 +\\
\quad Attenuanti × (-0.2)\\
)
}}

\vspace{0.5cm}
\textcolor{blue}{\textbf{Questo è un neurone artificiale!}}
\end{center}
\end{column}
\end{columns}

\end{frame}
%
%..................................................................
%
% Slide 4: Funzioni di Attivazione Spiegate Semplici
\begin{frame}
\frametitle{Le Funzioni di Attivazione}
\framesubtitle{Gli "Interruttori Intelligenti"}

\begin{columns}
\begin{column}{0.5\textwidth}
\textbf{Sigmoid - "L'Acceleratore"}
\begin{itemize}
    \item Come l'acceleratore di un'auto
    \item Più premi → più veloce vai
    \item Ma c'è un limite massimo!
    \item Output: da 0 a 1 (come percentuale)
\end{itemize}

\vspace{0.5cm}
\textbf{ReLU - "L'Interruttore"}
\begin{itemize}
    \item Come un interruttore della luce
    \item Se il segnale è debole → Spento (0)
    \item Se il segnale è forte → Acceso (il valore stesso)
    \item Semplice ma efficace!
\end{itemize}
\end{column}

\begin{column}{0.5\textwidth}
\begin{center}
\textbf{Analogia con l'Insegnamento:}

\vspace{0.3cm}
\fbox{\parbox{0.9\textwidth}{
\textit{Sigmoid}: Come dare voti da 0 a 10 - transizione graduale

\vspace{0.3cm}
\textit{ReLU}: Come dare Sufficiente/Insufficiente - soglia netta
}}

\vspace{0.5cm}
\textcolor{blue}{\textbf{Entrambi trasformano l'input in output utile!}}
\end{center}
\end{column}
\end{columns}

\end{frame}
%__________________________________________________________________________
%
\section{Apprendimento Supervisionato}
%
%..................................................................
%
\begin{frame}
\frametitle{Come Impara una Rete Neurale?}
\framesubtitle{L'Analogia del Docente Inesperto}

\begin{columns}
\begin{column}{0.5\textwidth}
\textbf{Immaginate un Docente alle Prime Armi:}

\vspace{0.3cm}
\textcolor{red}{\textbf{Situazione iniziale:}}
\begin{itemize}
    \item Non sa ancora valutare i compiti
    \item Ha criteri confusi e imprecisi
    \item Le sue valutazioni sono casuali
    \item Studenti e colleghi sono insoddisfatti
\end{itemize}

\vspace{0.3cm}
\textcolor{blue}{\textbf{Cosa serve per migliorare?}}
\begin{itemize}
    \item \textit{Supervisione} di un tutor esperto
    \item \textit{Correzione} degli errori
    \item \textit{Pratica} con molti esempi
    \item \textit{Feedback} costante
\end{itemize}
\end{column}

\begin{column}{0.5\textwidth}
\begin{center}
\textbf{Una Rete Neurale è Identica!}

\vspace{0.3cm}
\fbox{\parbox{0.9\textwidth}{
\centering
\textcolor{red}{\textbf{Inizialmente:}}\\
I pesi sono casuali\\
Le previsioni sono sbagliate\\

\vspace{0.3cm}
\textcolor{green}{\textbf{Con l'addestramento:}}\\
I pesi si aggiustano\\
Le previsioni migliorano\\
}}

\vspace{0.5cm}
\textcolor{blue}{\textbf{Chiave del successo:}} 
\textit{Apprendimento Supervisionato}
\end{center}
\end{column}
\end{columns}

\end{frame}
\begin{frame}
\frametitle{I Dati di Addestramento: I "Libri di Testo" della Rete}
\framesubtitle{Esempi con Risposte Corrette}

\begin{columns}
\begin{column}{0.6\textwidth}
\textbf{Cosa Servono alla Rete per Imparare:}

\vspace{0.3cm}
\textcolor{blue}{\textbf{Dataset di Training:}}
\begin{itemize}
    \item Migliaia (o milioni) di esempi
    \item Ogni esempio ha \textit{input} + \textit{output corretto}
    \item Qualità $>$ Quantità (ma servono entrambe!)
    \item Diversità negli esempi
\end{itemize}

\vspace{0.3cm}
\textbf{Analogia Scolastica:}
\begin{itemize}
    \item \textit{Input}: Testo di un problema di matematica
    \item \textit{Output corretto}: Soluzione step-by-step
    \item \textit{Addestramento}: Mostrare 1000 problemi risolti
    \item \textit{Test}: Dare un problema nuovo
\end{itemize}
\end{column}

\begin{column}{0.4\textwidth}
\begin{center}
\textbf{Esempio: Riconoscimento Email Spam}

\vspace{0.3cm}
\begin{tabular}{|c|c|}
\hline
\textbf{Email} & \textbf{Etichetta} \\
\hline
"Vinci 1000€!" & SPAM \\
\hline
"Riunione domani" & LEGITTIMA \\
\hline
"CLICCA QUI!!!" & SPAM \\
\hline
"Buon compleanno" & LEGITTIMA \\
\hline
\end{tabular}

\vspace{0.5cm}
\fbox{\parbox{0.9\textwidth}{
\centering
\textbf{La rete impara dai pattern:}\\
Molte maiuscole + punti esclamativi → SPAM
}}
\end{center}
\end{column}
\end{columns}
\end{frame}
%
%..................................................................
%
\begin{frame}
\frametitle{La "Matematica del Miglioramento"}
\framesubtitle{Come Funziona la Correzione dei Pesi}

\begin{columns}
\begin{column}{0.5\textwidth}
\textbf{Analogia: Aggiustare la Mira}

\vspace{0.3cm}
\textcolor{blue}{\textbf{Quando tirate una freccia:}}
\begin{itemize}
    \item Mirate al centro del bersaglio
    \item Vedete dove colpite
    \item Aggiustate la mira per la prossima freccia
    \item \textit{Se troppo a sinistra} → Mirate più a destra
    \item \textit{Se troppo in alto} → Mirate più in basso
\end{itemize}

\end{column}

\begin{column}{0.5\textwidth}
\textbf{Nella Rete Neurale:}

\vspace{0.3cm}
\textcolor{green}{\textbf{Formula Semplificata:}}
\begin{center}
Nuovo Peso = Vecchio Peso + Correzione
\end{center}

\vspace{0.3cm}
\textcolor{red}{\textbf{La Correzione dipende da:}}
\begin{itemize}
    \item \textit{Quanto} è grande l'errore
    \item \textit{Quanto} ha contribuito quel peso all'errore
    \item \textit{Velocità} di apprendimento (learning rate)
\end{itemize}

\end{column}
\end{columns}

\end{frame}
%
%..................................................................
%
\begin{frame}
\frametitle{La "Matematica del Miglioramento"}
\framesubtitle{Come Funziona la Correzione dei Pesi}

\begin{columns}
\begin{column}{0.5\textwidth}

\vspace{0.3cm}
\textbf{La Regola d'Oro:}
\begin{center}
\fbox{\parbox{0.9\textwidth}{
\centering
\textit{Correzione proporzionale all'errore}\\
\vspace{0.2cm}
Errore grande → Correzione grande\\
Errore piccolo → Correzione piccola
}}
\end{center}
\end{column}

\begin{column}{0.5\textwidth}
\textbf{Esempio Numerico:}
\begin{itemize}
    \item Peso attuale: 0.3
    \item Errore: La rete ha sbagliato molto
    \item Questo peso ha causato l'errore
    \item Correzione: -0.1
    \item Nuovo peso: 0.3 - 0.1 = 0.2
\end{itemize}
\end{column}
\end{columns}

\end{frame}
%__________________________________________________________________________
%
\section{Rappresentazioni Interne}
%
%..................................................................
%
\begin{frame}{Il Concetto di Rappresentazione}

\begin{block}{Definizione}
Una \textbf{rappresentazione} è il modo in cui i dati vengono codificati internamente dall'algoritmo per facilitare il compito da svolgere.
\end{block}
\vspace{0.5cm}
\begin{columns}
\begin{column}{0.5\textwidth}
\textbf{Rappresentazione Originale}
\begin{itemize}
\item Pixel dell'immagine
\item Parole del testo  
\item Note musicali
\end{itemize}
\end{column}
\begin{column}{0.5\textwidth}
\textbf{Rappresentazione Appresa}
\begin{itemize}
\item Bordi e forme
\item Concetti semantici
\item Armonie e ritmi
\end{itemize}
\end{column}
\end{columns}

\pause
\vspace{.3cm}
\begin{alertblock}{Intuizione Chiave}
Le reti neurali imparano a \textbf{trasformare} i dati in rappresentazioni più utili per il problema!
\end{alertblock}

\end{frame}
%
%..................................................................
%
\begin{frame}
\frametitle{Esempio: Dall'Immagine al Concetto}
\begin{center}
\includegraphics[scale=.5]{../05-pictures/dimensione-cognitiva-3_pic_10.png} 
\end{center}
\end{frame}
%
%..................................................................
%
\begin{frame}
\frametitle{Esempio: Dall'Immagine al Concetto}
\begin{center}
\includegraphics[scale=.5]{../05-pictures/dimensione-cognitiva-3_pic_11.png} 
\end{center}
\end{frame}
%
%..................................................................
%
\begin{frame}
\frametitle{Esempio: Dall'Immagine al Concetto}
\begin{center}
\includegraphics[scale=.35]{../05-pictures/dimensione-cognitiva-3_pic_12.png} 
\end{center}

\begin{block}{Trasformazione Progressiva}
\begin{enumerate}
\item \textbf{Layer 1}: Rileva bordi e texture dai pixel
\item \textbf{Layer 2}: Combina bordi in forme geometriche
\item \textbf{Output}: Riconosce oggetti dalle forme
\end{enumerate}
\end{block}

\end{frame}
%
%..................................................................
%
\begin{frame}{La Magia delle Rappresentazioni Intermedie}

\begin{block}{Esempio: Riconoscimento di Volti}
\textbf{Cosa impara ogni livello:}
\begin{itemize}
\item \textbf{Layer 1}: Bordi, linee, contrasti
\item \textbf{Layer 2}: Naso, occhi, bocca (parti del volto)
\item \textbf{Layer 3}: Configurazioni facciali
\item \textbf{Output}: ``È un volto'' o ``Non è un volto''
\end{itemize}
\end{block}

\pause

\begin{block}{Perché È Rivoluzionario?}
\begin{itemize}
\item La rete \textbf{scopre automaticamente} le caratteristiche rilevanti
\item Non dobbiamo più programmare manualmente ``cosa cercare''
\item Ogni layer costruisce su quello precedente
\item Rappresentazioni sempre più \textbf{astratte} e \textbf{significative}
\end{itemize}
\end{block}

\end{frame}
%
%..................................................................
%
\begin{frame}{Visualizzare le Rappresentazioni: Un Esperimento Mentale}

\begin{block}{Scenario: Classificazione di Animali}
Dati originali: 1000 immagini di cani e gatti (28x28 pixel = 784 dimensioni)
\end{block}
\vspace{1cm}
\begin{columns}
\begin{column}{0.5\textwidth}
\textbf{Spazio Originale (784D)}
\begin{itemize}
\item Ogni pixel = una dimensione
\item Dati molto ``sparsi''
\item Difficile trovare pattern
\end{itemize}
\end{column}
\begin{column}{0.5\textwidth}
\textbf{Rappresentazione Interna (3D)}
\begin{itemize}
\item La rete ``comprime'' in 3D
\item Cani e gatti si separano!
\item Pattern evidenti
\end{itemize}
\end{column}
\end{columns}
\end{frame}
%
%..................................................................
%
\begin{frame}{Visualizzare le Rappresentazioni: Un Esperimento Mentale}

\begin{block}{Scenario: Classificazione di Animali}
Dati originali: 1000 immagini di cani e gatti (28x28 pixel = 784 dimensioni)
\end{block}

\vspace{0.5cm}

\begin{center}
\begin{tikzpicture}[scale=0.7]
% 3D representation
\draw[->] (0,0) -- (3,0) node[right] {Dimensione 1};
\draw[->] (0,0) -- (0,3) node[above] {Dimensione 2};
\draw[->] (0,0) -- (-1.5,-1.5) node[below left] {Dimensione 3};

% Cani (blu)
\fill[blue] (0.5,0.5) circle (2pt);
\fill[blue] (0.8,0.8) circle (2pt);
\fill[blue] (0.3,0.7) circle (2pt);
\fill[blue] (0.6,0.4) circle (2pt);

% Gatti (rosso)  
\fill[red] (2.5,2.5) circle (2pt);
\fill[red] (2.2,2.8) circle (2pt);
\fill[red] (2.7,2.2) circle (2pt);
\fill[red] (2.4,2.6) circle (2pt);

\node[blue] at (0.5,0.2) {\small Cani};
\node[red] at (2.5,2.1) {\small Gatti};
\end{tikzpicture}
\end{center}

\end{frame}
%
%..................................................................
%
\begin{frame}{Rappresentazioni per Diversi Domini}

\begin{block}{Immagini: Convolutional Neural Networks (CNN)}
\begin{itemize}
\item \textbf{Rappresentazione}: Da pixel a feature maps
\item \textbf{Trasformazioni}: Convoluzione, pooling
\item \textbf{Risultato}: Gerarchia di pattern visivi
\end{itemize}
\end{block}

\begin{block}{Testo: Transformer e Word Embeddings}
\begin{itemize}
\item \textbf{Rappresentazione}: Da parole a vettori numerici  
\item \textbf{Trasformazioni}: Attention mechanisms
\item \textbf{Risultato}: Significato semantico e relazioni
\end{itemize}
\end{block}
\end{frame}
%
%=====================================================================
%
\end{document}
