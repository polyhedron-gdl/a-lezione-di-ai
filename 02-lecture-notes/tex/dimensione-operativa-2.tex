\documentclass[aspectratio=169]{beamer}
%
%=====================================================================
%
\usetheme{Madrid}
\usecolortheme{default}
%
%=====================================================================
%
% Pacchetti necessari
\usepackage[utf8]{inputenc}
\usepackage[italian]{babel}
\usepackage{amsmath}
\usepackage{amsfonts}
\usepackage{amssymb}
\usepackage{graphicx}
\usepackage{xcolor}
\usepackage{listings}
\usepackage{tikz}
\usepackage{tcolorbox}
%
%=====================================================================
%
% Configurazione per i box colorati
\newtcolorbox{examplebox}[1][]{
    colback=green!5!white,
    colframe=green!75!black,
    title=#1,
    fonttitle=\bfseries
}

\newtcolorbox{warningbox}[1][]{
    colback=red!5!white,
    colframe=red!75!black,
    title=#1,
    fonttitle=\bfseries
}

\newtcolorbox{tipbox}[1][]{
    colback=blue!5!white,
    colframe=blue!75!black,
    title=#1,
    fonttitle=\bfseries
}
%
%=====================================================================
%
\title{Dimensione Operativa \\ 2. Prompt Engineering}
\subtitle{Tecniche e Strategie per Ottimizzare l'Interazione con i Modelli AI}
\setbeamercovered{transparent} 
\author{Giovanni Della Lunga\\{\footnotesize giovanni.dellalunga@unibo.it}}
\institute{A lezione di Intelligenza Artificiale} 
\date{Siena - Giugno 2025} 
%
%=====================================================================
%
\begin{document}

% Slide titolo
\begin{frame}
    \titlepage
\end{frame}

% Indice
\begin{frame}{Indice}
    \tableofcontents
\end{frame}
%
%=====================================================================
%
\AtBeginSection[]
{
  \begin{frame}
  \vfill
  \centering
  \begin{beamercolorbox}[sep=8pt,center,shadow=true,rounded=true]{title}  	 	 	\usebeamerfont{title}\insertsectionhead\par%
  \end{beamercolorbox}
  \vfill
  \end{frame}
}
\AtBeginSubsection{\frame{\subsectionpage}}
%__________________________________________________________________________
%
\section{Introduzione al Prompt Engineering}
%
%..................................................................
%
\begin{frame}
\frametitle{Che cos'è il Prompt Engineering?}
\begin{itemize}[<+->]
    \item \textbf{Definizione}: L'arte e la scienza di progettare istruzioni efficaci per modelli di intelligenza artificiale
    \item \textbf{Obiettivo}: Ottenere risposte più precise, utili e appropriate dall'AI
    \item \textbf{Analogia}: Come dare indicazioni stradali chiare - più sono precise, migliore è il risultato
    \item \textbf{Importanza}: La differenza tra una risposta generica e una soluzione perfetta per le nostre esigenze
\end{itemize}

\pause
\begin{examplebox}[Esempio Pratico]
\textbf{Prompt generico}: "Spiegami la fotosintesi"\\
\textbf{Prompt ottimizzato}: "Spiega la fotosintesi a studenti di seconda media usando un linguaggio semplice, con un esempio pratico e una metafora facile da ricordare"
\end{examplebox}
\end{frame}
%
%..................................................................
%
\begin{frame}
\frametitle{Perché è Important per gli Educatori?}
\begin{columns}
\begin{column}{0.6\textwidth}
\textbf{Vantaggi per l'insegnamento:}
\begin{itemize}
    \item Preparazione rapida di materiali didattici
    \item Creazione di esercizi personalizzati
    \item Spiegazioni adattate al livello degli studenti
    \item Generazione di esempi creativi
    \item Supporto nella valutazione
\end{itemize}
\end{column}
\begin{column}{0.4\textwidth}
\begin{tipbox}[Ricorda]
L'AI è uno strumento potente, ma la qualità dell'output dipende dalla qualità dell'input che forniamo!
\end{tipbox}
\end{column}
\end{columns}
\end{frame}
%__________________________________________________________________________
%
\section{Obiettivi Principali del Prompt Engineering}
%
%..................................................................
%
\begin{frame}
\frametitle{I Quattro Obiettivi Fondamentali}
\begin{enumerate}[<+->]
\Large
    \item \textbf{Precisione}: Ottenere informazioni accurate e specifiche
    \item \textbf{Rilevanza}: Ricevere contenuti pertinenti al contesto
    \item \textbf{Chiarezza}: Avere spiegazioni comprensibili per il target
    \item \textbf{Completezza}: Coprire tutti gli aspetti necessari dell'argomento
\end{enumerate}
\end{frame}
%
%..................................................................
%
\begin{frame}
\centering
%\begin{center}
\begin{tikzpicture}
\node[draw, circle, fill=blue!20, minimum size=2cm] at (0,0) {\textbf{Prompt Perfetto}};
\node[draw, rectangle, fill=green!20] at (-2.5,1.5) {Preciso};
\node[draw, rectangle, fill=yellow!20] at (2.5,1.5) {Rilevante};
\node[draw, rectangle, fill=orange!20] at (-2.5,-1.5) {Chiaro};
\node[draw, rectangle, fill=red!20] at (2.5,-1.5) {Completo};
\draw[->] (-1.8,1.2) -- (-0.7,0.7);
\draw[->] (1.8,1.2) -- (0.7,0.7);
\draw[->] (-1.8,-1.2) -- (-0.7,-0.7);
\draw[->] (1.8,-1.2) -- (0.7,-0.7);
\end{tikzpicture}
%\end{center}
\end{frame}
%
%..................................................................
%
\begin{frame}
\frametitle{Esempio: Evoluzione di un Prompt}
\begin{examplebox}[Versione Base]
"Crea un quiz di matematica"
\end{examplebox}

\pause
\begin{examplebox}[Versione Migliorata]
"Crea un quiz di matematica per studenti di seconda media sulle frazioni, con 5 domande a risposta multipla di difficoltà crescente"
\end{examplebox}

\pause
\begin{examplebox}[Versione Ottimizzata]
"Crea un quiz di matematica per studenti di seconda media sulle frazioni. Includi 5 domande a risposta multipla con difficoltà crescente. Per ogni domanda fornisci: la spiegazione della risposta corretta, un suggerimento per risolvere problemi simili, e una breve spiegazione dell'errore più comune. Usa esempi pratici dalla vita quotidiana."
\end{examplebox}
\end{frame}
%__________________________________________________________________________
%
\section{Tecniche Fondamentali}
%
%..................................................................
%
\begin{frame}
\frametitle{1. Specificità e Contesto}
\textbf{Principio}: Più dettagli fornisci, migliori saranno i risultati
\vspace{1cm}
\begin{columns}
\begin{column}{0.5\textwidth}
\textbf{Prompt Vago:}
\begin{itemize}
    \item "Spiegami la storia"
    \item "Aiutami con la lezione"
    \item "Crea un esercizio"
\end{itemize}
\end{column}
\begin{column}{0.5\textwidth}
\textbf{Prompt Specifico:}
\begin{itemize}
    \item "Spiegami la Rivoluzione Francese per studenti di terza media"
    \item "Aiutami a creare una lezione di 45 minuti sui vulcani"
    \item "Crea 3 problemi di geometria sui triangoli per verifica"
\end{itemize}
\end{column}
\end{columns}
\end{frame}
%
%..................................................................
%
\begin{frame}
\vspace{1cm}
\large
\begin{tipbox}[Elementi da Specificare]
\textbf{Chi}: Target audience (età, livello)\\
\textbf{Cosa}: Argomento specifico\\
\textbf{Come}: Formato desiderato\\
\textbf{Perché}: Obiettivo/scopo
\end{tipbox}
\end{frame}
%
%..................................................................
%
\begin{frame}
\frametitle{2. Definizione del Ruolo}
\begin{itemize}
\normalsize
\item Un prompt che inizia con "Agisci come..." è estremamente utile per indirizzare l'intelligenza artificiale a rispondere o comportarsi in modo specifico, assumendo un ruolo o un punto di vista. 
\item Questo approccio permette di ottenere risposte più contestualizzate e rilevanti rispetto a un problema o a una situazione particolare. 
\item Permette all'AI di adattare il tono, il linguaggio e il livello di complessità in base alle esigenze specifiche; 
\item E' applicabile a diversi ambiti, come simulazioni di docenti, consulenti o esperti in una materia.
\end{itemize}
\end{frame}
%
%..................................................................
%
\begin{frame}
\frametitle{2. Definizione del Ruolo}
\textbf{Tecnica}: Assegna un ruolo specifico all'AI per ottenere risposte più mirate

\begin{examplebox}[Esempi di Ruoli]
\textbf{Per Matematica}: "Agisci come un insegnante di matematica esperto con 20 anni di esperienza nelle scuole medie..."

\textbf{Per Scienze}: "Sei un divulgatore scientifico che sa spiegare concetti complessi in modo semplice..."

\textbf{Per Letteratura}: "Comportati come un critico letterario che ama rendere la letteratura accessibile ai giovani..."
\end{examplebox}

\pause
\textbf{Perché funziona?}
\begin{itemize}
    \item L'AI adotta il linguaggio e l'approccio del ruolo
    \item Fornisce risposte più coerenti con l'expertise richiesta
    \item Mantiene uno stile appropriato per tutto l'output
\end{itemize}
\end{frame}
%
%..................................................................
%
\begin{frame}
\frametitle{3. Esempi e Modelli (Few-Shot Learning)}
\textbf{Principio}: Mostra all'AI il formato desiderato con esempi concreti

\begin{examplebox}[Esempio Pratico - Creazione Domande]
\textbf{Prompt}: "Crea domande di comprensione del testo seguendo questo formato:

\textbf{Esempio 1:}
Testo: 'Il cane corre nel parco'
Domanda: Dove corre il cane?
Risposta: Nel parco
Tipo: Comprensione letterale

Ora crea 3 domande simili per questo testo: 'Maria studia matematica in biblioteca ogni pomeriggio per prepararsi all'esame.'"
\end{examplebox}

\pause
\textbf{Vantaggi}:
\begin{itemize}
    \item Formato consistente
    \item Risultati prevedibili
    \item Facile da replicare
\end{itemize}
\end{frame}
%
%..................................................................
%
\begin{frame}
\frametitle{4. Istruzioni Strutturate}
\textbf{Tecnica}: Organizza le richieste in sezioni chiare

\begin{examplebox}[Modello di Prompt Strutturato]
\textbf{CONTESTO}: Stai preparando una lezione di scienze per studenti di prima media

\textbf{COMPITO}: Spiega il ciclo dell'acqua

\textbf{FORMATO}: 
- Introduzione accattivante
- 4 fasi principali con spiegazioni semplici
- Un'attività pratica
- 3 domande di verifica

\textbf{STILE}: Linguaggio adatto a ragazzi di 11-12 anni, usa esempi dalla vita quotidiana

\textbf{LUNGHEZZA}: Circa 300 parole per la spiegazione
\end{examplebox}
\end{frame}
%
%..................................................................
%
\begin{frame}
\frametitle{5. Raffinamento Iterativo}
\textbf{Approccio}: Migliora progressivamente il prompt attraverso tentativi successivi

\begin{center}
\begin{tikzpicture}[
  node distance = 4cm,      % più spazio fra i centri
  >=stealth,                % frecce più carine (facoltativo)
]
\node[draw, rectangle, fill=blue!20] (start) {Prompt Iniziale};
\node[draw, rectangle, fill=yellow!20, right of=start] (test) {Test Risultato};
\node[draw, rectangle, fill=green!20, right of=test] (refine) {Raffina Prompt};
\node[draw, rectangle, fill=red!20, right of=refine] (final) {Risultato Ottimale};

\draw[->] (start) -- (test);
\draw[->] (test) -- (refine);
\draw[->] (refine) -- (final);
\draw[->] (refine) to[bend left=60] (test);
\end{tikzpicture}
\end{center}

\pause
\begin{tipbox}[Strategia di Raffinamento]
\begin{enumerate}
    \item Inizia con un prompt semplice
    \item Analizza cosa manca o non funziona
    \item Aggiungi dettagli specifici
    \item Testa nuovamente
    \item Ripeti fino a soddisfazione
\end{enumerate}
\end{tipbox}
\end{frame}
%__________________________________________________________________________
%
\section{Esempi Pratici per l'Insegnamento}
%
%..................................................................
%
\begin{frame}
\frametitle{Esempio 1: Creazione di Materiale Didattico}
\textbf{Scenario}: Preparare una spiegazione dell'apparato respiratorio

\begin{examplebox}[Prompt Ottimizzato]
"Agisci come un insegnante di scienze esperto. Crea una spiegazione dell'apparato respiratorio per studenti di seconda media. 

\textbf{Struttura richiesta}:
1. Introduzione con una metafora efficace
2. Descrizione dei principali organi (polmoni, trachea, bronchi)
3. Spiegazione del processo di respirazione in 3 passaggi semplici
4. Un'analogia con oggetti familiari
5. 2 curiosità interessanti
6. Un esperimento semplice da fare in classe

\textbf{Linguaggio}: Adatto a ragazzi di 12-13 anni, evita termini troppo tecnici.
\textbf{Lunghezza}: 400-500 parole"
\end{examplebox}
\end{frame}
%
%..................................................................
%
\begin{frame}
\frametitle{Esempio 2: Creazione di Esercizi Personalizzati}
\textbf{Scenario}: Creare problemi di matematica adattati

\begin{examplebox}[Prompt per Esercizi]
"Sei un esperto creatore di esercizi di matematica. Crea 5 problemi sulle percentuali per studenti di seconda media con queste caratteristiche:

\textbf{Livello}: Progressivo (dal più facile al più difficile)
\textbf{Contesto}: Situazioni reali che interessano i ragazzi (sport, videogiochi, social media, shopping)
\textbf{Formato}: 
- Testo del problema
- Operazioni da svolgere
- Risultato
- Spiegazione passo-passo
- Un suggerimento per problemi simili

\textbf{Stile}: Coinvolgente e relazionabile per pre-adolescenti"
\end{examplebox}
\end{frame}
%
%..................................................................
%
\begin{frame}
\frametitle{Esempio 3: Supporto alla Valutazione}
\textbf{Scenario}: Creare criteri di valutazione chiari

\begin{examplebox}[Prompt per Rubrica]
"Agisci come un esperto in valutazione scolastica. Crea una rubrica di valutazione per un tema di antologia sul racconto di avventura.

\textbf{Parametri da valutare}:
- Contenuto e creatività
- Struttura narrativa
- Uso della lingua italiana
- Originalità

\textbf{Livelli}: Eccellente (10), Buono (8-9), Sufficiente (6-7), Insufficiente (4-5)

\textbf{Formato}: Tabella con descrittori specifici per ogni livello e suggerimenti per il miglioramento.

\textbf{Target}: Studenti di prima media, linguaggio comprensibile anche per gli studenti"
\end{examplebox}
\end{frame}
%__________________________________________________________________________
%
\section{Errori Comuni e Come Evitarli}
%
%..................................................................
%
\begin{frame}
\frametitle{I 5 Errori Più Comuni}
\begin{enumerate}[<+->]
    \item \textbf{Prompt troppo generici}
    \vspace{0.2cm}
    \begin{warningbox}[Sbagliato]
    "Aiutami con una lezione di storia"
    \end{warningbox}
    \begin{examplebox}[Corretto]
    "Aiutami a creare una lezione di 50 minuti sulla Seconda Guerra Mondiale per terza media, con focus sulle cause del conflitto"
    \end{examplebox}
    
    \item \textbf{Non specificare il target}
    \vspace{0.2cm}
    \begin{warningbox}[Sbagliato]
    "Spiega la fotosintesi"
    \end{warningbox}
    \begin{examplebox}[Corretto]
    "Spiega la fotosintesi a studenti di prima media usando un linguaggio semplice"
    \end{examplebox}
\end{enumerate}
\end{frame}
%
%..................................................................
%
\begin{frame}
\frametitle{Altri Errori da Evitare}
\begin{enumerate}
    \setcounter{enumi}{2}
    \item \textbf{Aspettarsi perfezione al primo tentativo}
    \begin{itemize}
        \item Il prompt engineering è un processo iterativo
        \item Testa e raffina progressivamente
    \end{itemize}
    
    \item \textbf{Non fornire esempi quando necessario}
    \begin{itemize}
        \item Per formati specifici, mostra sempre un esempio
        \item L'AI impara meglio vedendo il modello desiderato
    \end{itemize}
    
    \item \textbf{Dimenticare il contesto disciplinare}
    \begin{itemize}
        \item Specifica sempre la materia e il livello scolastico
        \item Indica eventuali prerequisiti o limitazioni
    \end{itemize}
\end{enumerate}

\pause
\begin{tipbox}[Consiglio Pratico]
Tieni un "diario dei prompt" con quelli che funzionano meglio per riutilizzarli!
\end{tipbox}
\end{frame}
%__________________________________________________________________________
%
\section{Strumenti e Suggerimenti Pratici}
%
%..................................................................
%
\begin{frame}
\frametitle{Template di Prompt per Docenti}
\begin{examplebox}[Modello Base Universale]
\textbf{RUOLO}: Sei un [insegnante di X materia] esperto

\textbf{COMPITO}: [Descrizione specifica del compito]

\textbf{TARGET}: Studenti di [classe] media ([età] anni)

\textbf{FORMATO}: 
- [Specifica struttura desiderata]
- [Lunghezza approssimativa]
- [Elementi obbligatori]

\textbf{STILE}: [Linguaggio adatto, tono, approccio]

\textbf{VINCOLI}: [Limitazioni, cose da evitare]

\textbf{OBIETTIVO}: [Cosa vuoi ottenere]
\end{examplebox}
\end{frame}
%
%..................................................................
%
\begin{frame}
\frametitle{Checklist per Prompt di Qualità}
\textbf{Prima di inviare il prompt, verifica:}

\begin{columns}
\begin{column}{0.5\textwidth}
\textbf{Contenuto}
\begin{itemize}
    \item[$\square$] Ruolo definito
    \item[$\square$] Compito specifico
    \item[$\square$] Target chiarito
    \item[$\square$] Formato specificato
    \item[$\square$] Esempi inclusi (se necessari)
\end{itemize}
\end{column}
\begin{column}{0.5\textwidth}
\textbf{Qualità}
\begin{itemize}
    \item[$\square$] Linguaggio appropriato
    \item[$\square$] Lunghezza indicata
    \item[$\square$] Vincoli specificati
    \item[$\square$] Obiettivo chiaro
    \item[$\square$] Contesto completo
\end{itemize}
\end{column}
\end{columns}

\pause
\vspace{1cm}
\begin{tipbox}[Regola d'oro]
Se un collega dovesse leggere il tuo prompt, capirebbe esattamente cosa vuoi ottenere?
\end{tipbox}
\end{frame}
%
%..................................................................
%
\begin{frame}
\frametitle{Suggerimenti per l'Uso Quotidiano}
\textbf{Integrate il prompt engineering nella vostra routine:}

\begin{itemize}[<+->]
    \item \textbf{Preparazione lezioni}: Create template per le vostre materie
    \item \textbf{Creazione esercizi}: Mantenete una raccolta di prompt testati
    \item \textbf{Valutazione}: Usate l'AI per creare rubriche e criteri
    \item \textbf{Differenziazione}: Adattate contenuti per diversi livelli
    \item \textbf{Creatività}: Brainstorming per attività innovative
\end{itemize}

\pause
\vspace{0.5cm}
\begin{warningbox}[Importante]
Ricordate sempre di:
\begin{itemize}
    \item Verificare la correttezza delle informazioni
    \item Adattare l'output alle vostre esigenze specifiche
    \item Mantenere il vostro tocco personale nell'insegnamento
\end{itemize}
\end{warningbox}
\end{frame}
%__________________________________________________________________________
%
\section{Conclusioni e Prospettive Future}
%
%..................................................................
%
\begin{frame}
\frametitle{Riassunto dei Punti Chiave}
\textbf{Il Prompt Engineering vi permette di:}
\begin{itemize}
    \item \textbf{Risparmiare tempo} nella preparazione del materiale didattico
    \item \textbf{Personalizzare} contenuti per i vostri studenti
    \item \textbf{Innovare} le metodologie didattiche
    \item \textbf{Supportare} la valutazione e il feedback
    \item \textbf{Stimolare} la creatività nell'insegnamento
\end{itemize}

\pause
\vspace{1cm}
\textbf{Ricordate sempre:}
\begin{center}
\Large
\textcolor{blue}{\textbf{La tecnologia è uno strumento,\\la pedagogia è l'arte!}}
\end{center}
\end{frame}
%
%..................................................................
%
\begin{frame}
\frametitle{Prospettive Future}
\textbf{L'evoluzione del prompt engineering nell'educazione:}

\begin{columns}
\begin{column}{0.6\textwidth}
\begin{itemize}
    \item \textbf{Personalizzazione avanzata}: AI che si adatta ai singoli studenti
    \item \textbf{Valutazione automatica}: Feedback immediato e dettagliato
    \item \textbf{Contenuti multimediali}: Integrazione di testo, immagini, video
    \item \textbf{Traduzione istantanea}: Supporto per studenti non italofoni
    \item \textbf{Accessibilità}: Adattamenti per diverse esigenze speciali
\end{itemize}
\end{column}
\begin{column}{0.4\textwidth}
\begin{tipbox}[Il Futuro]
L'AI diventerà un assistente sempre più sofisticato, ma il ruolo dell'insegnante rimarrà centrale nell'educazione!
\end{tipbox}
\end{column}
\end{columns}
\end{frame}
%
%..................................................................
%
\begin{frame}
\frametitle{Primi Passi Pratici}
\textbf{Come iniziare da domani:}

\begin{enumerate}
    \item \textbf{Scegliete una materia} di vostra competenza
    \item \textbf{Identificate un'esigenza} specifica (es: spiegazione difficile)
    \item \textbf{Create il vostro primo prompt} usando i template mostrati
    \item \textbf{Testate e raffinate} il risultato
    \item \textbf{Documentate} cosa funziona per riutilizzarlo
\end{enumerate}

\pause
\vspace{1cm}
\begin{examplebox}[Sfida Personale]
Impegnatevi a creare e testare almeno 3 prompt diversi questa settimana. Condividete i risultati con i colleghi!
\end{examplebox}
\end{frame}
%
%..................................................................
%
\begin{frame}
\frametitle{Risorse e Approfondimenti}
\textbf{Per continuare ad approfondire:}

\begin{itemize}
    \item \textbf{Comunità online}: Gruppi di docenti che condividono prompt
    \item \textbf{Formazione continua}: Webinar e corsi specifici
    \item \textbf{Sperimentazione}: La pratica è il miglior insegnante
    \item \textbf{Collaborazione}: Lavorate in team con i colleghi
\end{itemize}
\end{frame}
%
%=====================================================================
%
\end{document}
%
%=====================================================================
%
