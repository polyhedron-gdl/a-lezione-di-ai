\documentclass[aspectratio=169]{beamer}
\usepackage[utf8]{inputenc}
\usepackage[italian]{babel}
\usepackage{amsmath}
\usepackage{graphicx}
\usepackage{tikz}
\usepackage{xcolor}
%
%=====================================================================
%
% Tema e colori
\usetheme{Madrid}
\usecolortheme{seahorse}
%
%=====================================================================
%
% Definizione colori personalizzati
\definecolor{aiblue}{RGB}{30,144,255}
\definecolor{mlgreen}{RGB}{46,139,87}
\definecolor{datacolor}{RGB}{255,140,0}
%
%=====================================================================
%
\title{Dimensione Cognitiva \\ 2. Macchine che Imparano}
\subtitle{Introduzione al Machine Learning}
\setbeamercovered{transparent} 
\author{Giovanni Della Lunga\\{\footnotesize giovanni.dellalunga@unibo.it}}
\institute{A lezione di Intelligenza Artificiale} 
\date{Siena - Giugno 2025} 
%
%=====================================================================
%
\begin{document}

% Slide titolo
\begin{frame}
    \titlepage
\end{frame}

% Indice
\begin{frame}{Indice}
    \tableofcontents
\end{frame}
%
%=====================================================================
%
\AtBeginSection[]
{
  %\begin{frame}<beamer>
  %\footnotesize	
  %\frametitle{Outline}
  %\begin{multicols}{2}
  %\tableofcontents[currentsection]
  %\end{multicols}	  
  %\normalsize
  %\end{frame}
  \begin{frame}
  \vfill
  \centering
  \begin{beamercolorbox}[sep=8pt,center,shadow=true,rounded=true]{title}  	 	 	\usebeamerfont{title}\insertsectionhead\par%
  \end{beamercolorbox}
  \vfill
  \end{frame}
}
\AtBeginSubsection{\frame{\subsectionpage}}
%__________________________________________________________________________
%
\section{Algoritmi Tradizionali vs Machine Learning}
%
%..................................................................
%
\begin{frame}
\frametitle{Algoritmo Tradizionale: Dalle Regole al Codice}
\begin{block}{Definizione}
Un algoritmo tradizionale è una sequenza di istruzioni precise e predefinite che il computer esegue passo dopo passo per risolvere un problema specifico.
\end{block}

\textbf{Caratteristiche:}
\begin{itemize}
\item Le regole sono scritte esplicitamente dal programmatore
\item Il comportamento è completamente spiegabile e prevedibile
\item Infatti la logica è trasparente e verificabile
\end{itemize}

\textbf{Esempio concreto - Calcolo dello sconto:}
\begin{itemize}
\item SE il cliente spende più di 100€ ALLORA applica sconto 10\%
\item ALTRIMENTI nessuno sconto
\item Il programmatore ha definito esattamente quando e come applicare lo sconto
\end{itemize}
\end{frame}
%
%..................................................................
%
\begin{frame}
\frametitle{Algoritmo Tradizionale: Dalle Regole al Codice}
\begin{center}
\includegraphics[scale=.7]{../05-pictures/dimensione-cognitiva-2_pic_0.png} 
\end{center}
\end{frame}
%
%..................................................................
%
\begin{frame}
\frametitle{Algoritmo Tradizionale: Dalle Regole al Codice}
\begin{center}
\includegraphics[scale=.6]{../05-pictures/dimensione-cognitiva-2_pic_1.png} 
\end{center}
\end{frame}
%
%..................................................................
%
\begin{frame}
\frametitle{Un gioco più complicato ...}
\begin{center}
\includegraphics[scale=.4]{../05-pictures/dimensione-cognitiva-2_pic_2.png} 
\end{center}
\end{frame}
%
%..................................................................
%
\begin{frame}
\frametitle{Machine Learning: Apprendimento dai Dati}
\begin{block}{Definizione}
Il Machine Learning è un approccio in cui l'algoritmo scopre automaticamente le regole analizzando grandi quantità di dati, senza che queste regole vengano programmate esplicitamente.
\end{block}

\textbf{Caratteristiche:}
\begin{itemize}
\item Le regole emergono dall'analisi dei dati
\item Il comportamento può variare in base ai dati di addestramento
\item L'algoritmo può gestire situazioni non previste dal programmatore
\item La logica interna è spesso complessa e non direttamente interpretabile
\end{itemize}

\textbf{Esempio concreto - Rilevamento frodi:}
\begin{itemize}
\item L'algoritmo analizza milioni di transazioni passate
\item Identifica automaticamente pattern sospetti
\item Impara a distinguere transazioni normali da quelle fraudolente
\item Non esistono regole esplicite scritte dal programmatore
\end{itemize}
\end{frame}
%
%..................................................................
%
\begin{frame}
\frametitle{Confronto Diretto: Riconoscimento di Spam}
\begin{columns}
\begin{column}{0.48\textwidth}
\textbf{\color{red}Approccio Tradizionale}
\begin{itemize}
\item Il programmatore scrive regole:
\begin{itemize}
\item Se contiene "GRATIS" → spam
\item Se ha più di 5 punti esclamativi → spam
\item Se mittente sconosciuto → spam
\end{itemize}
\item Ogni regola è esplicita
\item Facile da capire ma limitato
\item Non si adatta a nuovi tipi di spam
\end{itemize}
\end{column}
\begin{column}{0.48\textwidth}
\textbf{\color{aiblue}Machine Learning}
\begin{itemize}
\item L'algoritmo analizza:
\begin{itemize}
\item 100.000 email spam
\item 100.000 email legittime
\end{itemize}
\item Scopre automaticamente pattern:
\begin{itemize}
\item Frequenza di certe parole
\item Struttura del testo
\item Caratteristiche del mittente
\end{itemize}
\item Si adatta a nuovi tipi di spam
\end{itemize}
\end{column}
\end{columns}
\end{frame}
%__________________________________________________________________________
%
\section{Tutto comincia con una Retta...}
%
%..................................................................
%
\begin{frame}
    \centering
    \includegraphics[width=\paperwidth,height=\paperheight,keepaspectratio]{../05-pictures/dimensione-cognitiva-2_pic_3.png}
\end{frame}
%
%..................................................................
%
\begin{frame}{Il Problema: Valutare un Appartamento}
\begin{block}{Esempio pratico}
Come può un computer determinare il prezzo di un appartamento?
\begin{itemize}
    \item Analizzando migliaia di vendite passate
    \item Identificando le caratteristiche che influenzano il prezzo
    \item Creando un modello predittivo
\end{itemize}
\end{block}
\end{frame}
%
%..................................................................
%
\begin{frame}{Il Problema: Valutare un Appartamento}
\begin{columns}
\begin{column}{0.5\textwidth}
\textbf{Caratteristiche dell'appartamento:}
\begin{itemize}
    \item Superficie (m²)
    \item Numero di stanze
    \item Piano
    \item Zona della città
    \item Età dell'edificio
\end{itemize}
\end{column}
\begin{column}{0.5\textwidth}
\textbf{Domanda:}
\begin{center}
\Large
Quanto vale questo appartamento?
\end{center}

\vspace{0.5cm}

\textbf{Approccio tradizionale:}
\begin{itemize}
    \item Perizia manuale
    \item Confronto con vendite simili
    \item Esperienza dell'agente
\end{itemize}
\end{column}
\end{columns}
\end{frame}%
%..................................................................
%
\begin{frame}{Soluzione: Regressione Lineare}
\textbf{Idea base:} Trovare una relazione matematica tra caratteristiche e prezzo

\vspace{0.5cm}

\begin{block}{Modello Semplificato (una variabile)}
$$\text{Prezzo} = a \times \text{Superficie} + b$$
\end{block}

\vspace{0.5cm}

\begin{block}{Modello Completo (più variabili)}
$$\text{Prezzo} = a_1 \times \text{Superficie} + a_2 \times \text{Stanze} + a_3 \times \text{Piano} + b$$
\end{block}

\vspace{0.5cm}

\textbf{L'algoritmo impara:}
\begin{itemize}
    \item I \textbf{coefficienti} $a_1, a_2, a_3, \ldots$ (quanto influisce ogni caratteristica)
    \item L'\textbf{intercetta} $b$ (prezzo base)
\end{itemize}
\end{frame}
%
%..................................................................
%
\begin{frame}{Come Funziona l'Apprendimento?}
\begin{enumerate}
    \item \textbf{Training}: Raccogliamo dati di appartamenti già venduti
    \begin{itemize}
        \item Es: 90m², 3 stanze, 2° piano → venduto a €180.000
    \end{itemize}
    
    \vspace{0.3cm}
    
    \item \textbf{Algoritmo}: Trova la retta che meglio approssima i dati
    \begin{itemize}
        \item Minimizza l'errore tra prezzi reali e predetti
    \end{itemize}
    
    \vspace{0.3cm}
    
    \item \textbf{Predizione}: Utilizziamo il modello per nuovi appartamenti
    \begin{itemize}
        \item Es: 75m², 2 stanze, 1° piano → prezzo stimato?
    \end{itemize}
\end{enumerate}

\vspace{0.5cm}

\begin{alertblock}{Esempio Numerico}
Se il modello impara: Prezzo = 2000 × Superficie + 15000 × Stanze + 5000\\
Per 75m², 2 stanze: Prezzo = 2000×75 + 15000×2 + 5000 = \textbf{€185.000}
\end{alertblock}
\end{frame}
%
%..................................................................
%
\begin{frame}{Gli Ingredienti Fondamentali del Machine Learning}
\begin{block}{Un esempio semplice, ma completo!}
Sebbene l'esempio della regressione lineare sia molto semplice, esso contiene \textbf{tutti} gli ingredienti fondamentali del Machine Learning:
\end{block}
\begin{center}
\includegraphics[scale=.4]{../05-pictures/dimensione-cognitiva-2_pic_4.png} 
\end{center}
\end{frame}
%
%..................................................................
%
\begin{frame}{Gli Ingredienti Fondamentali del Machine Learning}
\textbf{Data-Driven (Guidato dai Dati)}
\begin{itemize}
    \item Le decisioni non sono programmate manualmente
    \item Il modello impara direttamente dai dati storici di vendita
\end{itemize}
\begin{center}
\includegraphics[scale=.4]{../05-pictures/dimensione-cognitiva-2_pic_4.png} 
\end{center}
\end{frame}
%
%..................................................................
%
\begin{frame}{Gli Ingredienti Fondamentali del Machine Learning}
\textbf{Funzione di Errore}
\begin{itemize}
    \item Misuriamo quanto le nostre predizioni si discostano dalla realtà
    \item Es: Errore Quadratico Medio (MSE)
\end{itemize}
\begin{center}
\includegraphics[scale=.4]{../05-pictures/dimensione-cognitiva-2_pic_4.png} 
\end{center}
\end{frame}
%
%..................................................................
%
\begin{frame}{Gli Ingredienti Fondamentali del Machine Learning}
\textbf{Metodo di Ottimizzazione}
\begin{itemize}
    \item Algoritmo che cerca i parametri che minimizzano l'errore
    \item Es: Gradient Descent, Least Squares
\end{itemize}
\begin{center}
\includegraphics[scale=.4]{../05-pictures/dimensione-cognitiva-2_pic_4.png} 
\end{center}
\end{frame}
%__________________________________________________________________________
%
\section{Il Processo di Addestramento}
%
%..................................................................
%
\begin{frame}
\frametitle{Addestramento: Definizione Tecnica}
\begin{block}{Che cos'è l'addestramento}
L'addestramento è il processo computazionale attraverso cui un algoritmo di machine learning analizza un dataset per identificare pattern statistici e costruire un modello matematico capace di fare predizioni su dati nuovi.
\end{block}

\textbf{Componenti essenziali:}
\begin{enumerate}
\item \textbf{Dataset di addestramento:} Insieme di esempi con input e output desiderati
\item \textbf{Algoritmo di apprendimento:} Procedura matematica che trova i pattern
\item \textbf{Funzione di costo:} Misura quanto l'algoritmo sbaglia
\item \textbf{Ottimizzazione:} Processo per ridurre gli errori
\end{enumerate}
\end{frame}
%
%..................................................................
%
\begin{frame}
\frametitle{Fasi del Processo di Addestramento}
\textbf{1. Inizializzazione}
\begin{itemize}
\item L'algoritmo inizia con parametri casuali
\item Non sa ancora come risolvere il problema
\end{itemize}

\textbf{2. Presentazione dei dati}
\begin{itemize}
\item L'algoritmo riceve un esempio dal dataset
\item Prova a fare una predizione con i parametri attuali
\end{itemize}

\textbf{3. Calcolo dell'errore}
\begin{itemize}
\item Confronta la sua predizione con la risposta corretta
\item Calcola numericamente quanto ha sbagliato
\end{itemize}

\textbf{4. Aggiornamento dei parametri}
\begin{itemize}
\item Modifica leggermente i suoi parametri interni
\item L'obiettivo è ridurre l'errore per esempi simili
\end{itemize}

\textbf{5. Iterazione}
\begin{itemize}
\item Ripete il processo per tutti gli esempi nel dataset
\item Continua per molti cicli (epoche) fino a convergenza
\end{itemize}
\end{frame}
%
%..................................................................
%
\begin{frame}
\frametitle{Esempio Dettagliato: Predizione Prezzi Case}
\textbf{Dataset:} 10.000 case con caratteristiche e prezzi reali

\textbf{Input per ogni casa:}
\begin{itemize}
\item Superficie (mq), Numero stanze, Età, Distanza dal centro
\end{itemize}

\textbf{Output:} Prezzo di vendita

\textbf{Processo di addestramento:}
\begin{enumerate}
\item L'algoritmo inizia con una formula con parametri casuali: 
$$Prezzo = a \times Superficie + b \times Stanze + c \times Eta + d$$
\item Per la prima casa (100mq, 3 stanze, 10 anni): predice 150.000 EUR
\item Il prezzo reale era 200.000 EUR → errore di 50.000 EUR
\item Aggiusta i coefficienti $a, b, c, d$ per ridurre questo errore
\item Ripete per tutte le 10.000 case
\item Dopo molte iterazioni, la formula diventa accurata
\end{enumerate}
\end{frame}
%
%..................................................................
%
\begin{frame}
\frametitle{Un po' di Lessico: Features e Labels}
\begin{center}
\includegraphics[scale=.5]{../05-pictures/dimensione-cognitiva-2_pic_5.png} 
\end{center}
\end{frame}
%__________________________________________________________________________
%
\section{Dalla Regressione alla Classificazione}
%
%..................................................................
%
\begin{frame}
\frametitle{Dalla Regressione alla Classificazione}
\begin{center}
\includegraphics[scale=.5]{../05-pictures/dimensione-cognitiva-2_pic_6.png} 
\end{center}
\end{frame}
%
%..................................................................
%
\begin{frame}
\frametitle{Dalla Regressione alla Classificazione}
\begin{center}
\includegraphics[scale=.5]{../05-pictures/dimensione-cognitiva-2_pic_7.png} 
\end{center}
\end{frame}
%
%..................................................................
%
\begin{frame}{Dalla Regressione alla Classificazione}
\begin{columns}
\begin{column}{0.45\textwidth}
\textbf{Regressione}
\begin{itemize}
    \item Predice valori \textbf{continui}
    \item Output: numeri reali
    \item Esempio: prezzo €185.000
\end{itemize}

\vspace{0.3cm}

\begin{block}{Formula}
$$y = w_1 x_1 + w_2 x_2 + b$$
dove $y \in \mathbb{R}$
\end{block}
\end{column}

\begin{column}{0.45\textwidth}
\textbf{Classificazione}
\begin{itemize}
    \item Predice \textbf{categorie}
    \item Output: classi discrete
    \item Esempio: "Spam" o "Non Spam"
\end{itemize}

\vspace{0.3cm}

\begin{block}{Stesso Principio!}
$$\text{Decisione} = f(w_1 x_1 + w_2 x_2 + b)$$
dove $f$ trasforma in categorie
\end{block}
\end{column}
\end{columns}

\vspace{0.5cm}

\begin{alertblock}{Concetto Chiave}
Anche nella classificazione cerchiamo i \textbf{pesi ottimali} $w_1, w_2, \ldots$ che minimizzano l'errore!
\end{alertblock}
\end{frame}
%
%..................................................................
%
\begin{frame}
\frametitle{Che cos'è una funzione?}
\begin{center}
\Large
Una funzione è come un \textbf{operatore}:\\[0.5cm]
\begin{tikzpicture}
\draw[thick, rounded corners] (0,0) rectangle (4,2);
\node at (2,1) {\Large FUNZIONE};
\draw[->, thick, blue] (-1.5,1) -- (0,1);
\node[left] at (-1.5,1) {\textbf{Input}};
\draw[->, thick, red] (4,1) -- (5.5,1);
\node[right] at (5.5,1) {\textbf{Output}};
\end{tikzpicture}
\end{center}

\vspace{0.5cm}
\begin{itemize}
\item Riceve un \textcolor{blue}{\textbf{input}} (dato di ingresso)
\item Esegue una \textbf{trasformazione}
\item Produce un \textcolor{red}{\textbf{output}} (risultato)
\end{itemize}
\end{frame}
%
%..................................................................
%
\begin{frame}
\frametitle{Che cos'è una funzione?}

\begin{center}
\Large
$f(x) = y$
\end{center}

\vspace{0.5cm}
\begin{itemize}
\item \textbf{$f$} è il nome della funzione
\item \textbf{$x$} è l'input (variabile indipendente)
\item \textbf{$y$} è l'output (variabile dipendente)
\end{itemize}

\vspace{0.5cm}
\textbf{Esempio semplice:}
$$f(x) = 2x + 1$$

\begin{itemize}
\item Se $x = 3$, allora $f(3) = 2 \cdot 3 + 1 = 7$
\item Se $x = 0$, allora $f(0) = 2 \cdot 0 + 1 = 1$
\end{itemize}
\end{frame}
%
%..................................................................
%
\begin{frame}{Dalla Regressione alla Classificazione}
\begin{center}
\includegraphics[scale=.6]{../05-pictures/dimensione-cognitiva-2_pic_8.png} 
\end{center}
\end{frame}
%
%..................................................................
%
\begin{frame}{Dalla Regressione alla Classificazione}
\begin{center}
\includegraphics[scale=.5]{../05-pictures/dimensione-cognitiva-2_pic_9.png} 
\end{center}
\end{frame}
%
%..................................................................
%
\begin{frame}{Esempio: Rilevamento Email Spam}
\textbf{Problema}: Classificare automaticamente le email come "Spam" o "Non Spam"

\vspace{0.5cm}

\begin{block}{Caratteristiche dell'Email (Features)}
\begin{itemize}
    \item $x_1$: Numero di parole "GRATIS"
    \item $x_2$: Numero di punti esclamativi
    \item $x_3$: Presenza di link sospetti (0 o 1)
    \item $x_4$: Lunghezza dell'email
\end{itemize}
\end{block}

\vspace{0.5cm}

\begin{block}{Modello Lineare}
$$\text{Score} = w_1 \cdot x_1 + w_2 \cdot x_2 + w_3 \cdot x_3 + w_4 \cdot x_4 + b$$
\end{block}

\end{frame}
%
%..................................................................
%
\begin{frame}{Esempio: Rilevamento Email Spam}
\textbf{Problema}: Classificare automaticamente le email come "Spam" o "Non Spam"

\vspace{0.5cm}

\begin{block}{Caratteristiche dell'Email (Features)}
\begin{itemize}
    \item $x_1$: Numero di parole "GRATIS"
    \item $x_2$: Numero di punti esclamativi
    \item $x_3$: Presenza di link sospetti (0 o 1)
    \item $x_4$: Lunghezza dell'email
\end{itemize}
\end{block}


\vspace{0.3cm}

\textbf{Regola di Decisione}:
\begin{itemize}
    \item Se Score $> 0$ → \textcolor{red}{\textbf{SPAM}}
    \item Se Score $\leq 0$ → \textcolor{green}{\textbf{NON SPAM}}
\end{itemize}
\end{frame}
%
%..................................................................
%
\begin{frame}{I Pesi Raccontano una Storia}
\textbf{Supponiamo che l'algoritmo impari questi pesi:}

\vspace{0.5cm}

\begin{block}{Modello Appreso}
$$\text{Score} = \textcolor{red}{+3.2} \cdot x_1 + \textcolor{red}{+1.8} \cdot x_2 + \textcolor{red}{+5.1} \cdot x_3 + \textcolor{blue}{-0.01} \cdot x_4 + 0.5$$
\end{block}

\vspace{0.5cm}

\textbf{Interpretazione dei Pesi:}
\begin{itemize}
    \item $w_1 = +3.2$: Ogni "GRATIS" aumenta molto la probabilità di spam
    \item $w_2 = +1.8$: I punti esclamativi sono indicatori di spam
    \item $w_3 = +5.1$: I link sospetti sono il segnale più forte di spam
    \item $w_4 = -0.01$: Email più lunghe tendono a essere meno spam
\end{itemize}

\end{frame}
%
%..................................................................
%
\begin{frame}{I Pesi Raccontano una Storia}
\textbf{Supponiamo che l'algoritmo impari questi pesi:}

\vspace{0.5cm}

\begin{block}{Modello Appreso}
$$\text{Score} = \textcolor{red}{+3.2} \cdot x_1 + \textcolor{red}{+1.8} \cdot x_2 + \textcolor{red}{+5.1} \cdot x_3 + \textcolor{blue}{-0.01} \cdot x_4 + 0.5$$
\end{block}

\vspace{0.5cm}

\begin{alertblock}{Esempio Concreto}
Email con: 2 "GRATIS", 5 "!", 1 link sospetto, 200 parole\\
Score = $3.2 \times 2 + 1.8 \times 5 + 5.1 \times 1 - 0.01 \times 200 + 0.5 = 18.9 > 0$ → \textcolor{red}{\textbf{SPAM}}
\end{alertblock}
\end{frame}
%
%..................................................................
%
\begin{frame}{Come Trovare i Pesi Ottimali?}
\textbf{Stesso processo della regressione, ma con funzione di errore diversa!}

\vspace{0.5cm}

\begin{enumerate}
    \item \textbf{Dati di Training}
    \begin{itemize}
        \item Migliaia di email già etichettate: (features, label)
        \item Es: ([2, 5, 1, 200], "Spam"), ([0, 1, 0, 50], "Non Spam")
    \end{itemize}
    
    \vspace{0.3cm}
    
    \item \textbf{Funzione di Errore}
    \begin{itemize}
        \item Non più errore quadratico, ma \textbf{Cross-Entropy Loss}
        \item Penalizza classificazioni sbagliate
    \end{itemize}
    
    \vspace{0.3cm}
    
    \item \textbf{Ottimizzazione}
    \begin{itemize}
        \item Gradient Descent (come nella regressione!)
        \item Cerca i pesi $w_1, w_2, w_3, w_4, b$ che minimizzano l'errore
    \end{itemize}
\end{enumerate}

\vspace{0.5cm}
\end{frame}
%
%..................................................................
%
\begin{frame}{L'Unità Fondamentale del Machine Learning}
\begin{block}{Regressione e Classificazione: Stessa Filosofia}
Entrambi i problemi seguono lo stesso schema fondamentale:
\end{block}

\vspace{0.5cm}

\begin{columns}
\begin{column}{0.3\textwidth}
\centering
\textbf{1. Dati}\\
\vspace{0.2cm}
Input + Output\\
di training
\end{column}
\begin{column}{0.05\textwidth}
\centering
$\rightarrow$
\end{column}
\begin{column}{0.3\textwidth}
\centering
\textbf{2. Modello}\\
\vspace{0.2cm}
Combinazione lineare\\
$\sum w_i x_i + b$
\end{column}
\begin{column}{0.05\textwidth}
\centering
$\rightarrow$
\end{column}
\begin{column}{0.3\textwidth}
\centering
\textbf{3. Ottimizzazione}\\
\vspace{0.2cm}
Trova i pesi $w_i$\\
migliori
\end{column}
\end{columns}

\vspace{1cm}

\begin{alertblock}{Differenze Principali}
\begin{itemize}
    \item \textbf{Regressione}: Output continuo, Errore Quadratico
    \item \textbf{Classificazione}: Output discreto, Cross-Entropy Loss
    \item \textbf{Entrambi}: Cercano pesi ottimali Minimizzando l'Errore!
\end{itemize}
\end{alertblock}
\end{frame}
%__________________________________________________________________________
%
\section{Un Salto Concettuale: Dai Numeri alla Geometria}
%
%..................................................................
%
\begin{frame}{I Dati Come Punti nello Spazio}
\textbf{Intuizione Chiave}: Ogni dato può essere rappresentato come un \textbf{punto} in uno spazio multidimensionale

\vspace{0.5cm}

\begin{block}{Esempio: Email Spam (2 caratteristiche)}
\begin{itemize}
    \item Email A: 3 "GRATIS", 5 "!" → Punto $(3, 5)$
    \item Email B: 0 "GRATIS", 1 "!" → Punto $(0, 1)$  
    \item Email C: 8 "GRATIS", 12 "!" → Punto $(8, 12)$
\end{itemize}
\end{block}

\end{frame}
%
%..................................................................
%
\begin{frame}{I Dati Come Punti nello Spazio}
\textbf{Intuizione Chiave}: Ogni dato può essere rappresentato come un \textbf{punto} in uno spazio multidimensionale

\vspace{0.5cm}

\begin{columns}
\begin{column}{0.5\textwidth}
\textbf{Spazio 2D}
\begin{center}
\begin{tikzpicture}[scale=0.6]
\draw[->] (0,0) -- (4,0) node[right] {Num. "GRATIS"};
\draw[->] (0,0) -- (0,4) node[above] {Num. "!"};
%\draw[grid] (0,0) grid (3.5,3.5);
\draw[step=1cm, gray, very thin] (0,0) grid (3.5,3.5);

% Email points
\fill[red] (0.5,0.5) circle (3pt) node[below left] {\tiny Non Spam};
\fill[red] (1,1.5) circle (3pt);
\fill[blue] (2.5,2.5) circle (3pt) node[above right] {\tiny Spam};
\fill[blue] (3,3) circle (3pt);
\end{tikzpicture}
\end{center}
\end{column}

\begin{column}{0.5\textwidth}
\textbf{Osservazione}
\begin{itemize}
    \item \textcolor{red}{Email non-spam} tendono a raggrupparsi in una zona
    \item \textcolor{blue}{Email spam} si raggruppano in un'altra zona
    \item Esiste una \textbf{separazione naturale}!
\end{itemize}
\end{column}
\end{columns}
\end{frame}
%
%..................................................................
%
\begin{frame}{Generalizzazione a N Dimensioni}
\textbf{Il principio si estende a qualsiasi numero di caratteristiche!}

\vspace{0.5cm}

\begin{block}{Email con 4 Caratteristiche}
Ogni email diventa un punto in uno spazio a 4 dimensioni:
$$\text{Email} = (x_1, x_2, x_3, x_4) \in \mathbb{R}^4$$
dove:
\begin{itemize}
    \item $x_1$ = Numero "GRATIS"
    \item $x_2$ = Numero "!"  
    \item $x_3$ = Link sospetti (0/1)
    \item $x_4$ = Lunghezza email
\end{itemize}
\end{block}

\end{frame}
%
%..................................................................
%
\begin{frame}{Generalizzazione a N Dimensioni}
\textbf{Il principio si estende a qualsiasi numero di caratteristiche!}

\vspace{0.5cm}

\begin{columns}
\begin{column}{0.45\textwidth}
\textbf{Esempi Concreti}
\begin{itemize}
    \item Spam: $(2, 5, 1, 200)$
    \item Non Spam: $(0, 1, 0, 150)$
    \item Spam: $(4, 8, 1, 80)$
\end{itemize}
\end{column}
\begin{column}{0.45\textwidth}
\begin{alertblock}{Idea Fondamentale}
Anche se non possiamo \textbf{visualizzare} 4 dimensioni, il computer può \textbf{lavorare} in questo spazio!
\end{alertblock}
\end{column}
\end{columns}
\end{frame}
%
%..................................................................
%
\begin{frame}{Il Problema Diventa Geometrico}
\textbf{Classificazione} = \textbf{Separazione geometrica} nello spazio delle caratteristiche

\vspace{0.5cm}

\begin{center}
\begin{tikzpicture}[scale=0.8]
% Assi
\draw[->] (0,0) -- (5,0) node[right] {$x_1$ (caratteristica 1)};
\draw[->] (0,0) -- (0,4) node[above] {$x_2$ (caratteristica 2)};

% Punti spam (blu)
\fill[blue] (3.5,3) circle (3pt);
\fill[blue] (4,2.5) circle (3pt);
\fill[blue] (3,3.5) circle (3pt);
\fill[blue] (4.2,3.2) circle (3pt);

% Punti non-spam (rossi)
\fill[red] (1,1) circle (3pt);
\fill[red] (1.5,0.8) circle (3pt);
\fill[red] (0.8,1.5) circle (3pt);
\fill[red] (1.2,1.8) circle (3pt);

% Linea di separazione
\draw[thick, green] (2,0.5) -- (2.8,3.8);

% Etichette
\node[blue] at (4.5,2.8) {\textbf{Spam}};
\node[red] at (0.5,2) {\textbf{Non Spam}};
\node[green] at (3.2,2) {\textbf{Confine}};
\end{tikzpicture}
\end{center}

\vspace{0.5cm}

\end{frame}
%
%..................................................................
%
\begin{frame}{Il Problema Diventa Geometrico}
\textbf{Classificazione} = \textbf{Separazione geometrica} nello spazio delle caratteristiche
\vspace{0.5cm}
\begin{block}{Obiettivo del Machine Learning}
Trovare il \textbf{confine ottimale} che separa al meglio le due classi
\end{block}
\vspace{0.5cm}
\begin{itemize}
    \item \textbf{Confine lineare}: una retta (2D), un piano (3D), un iperpiano (N-D)
    \item \textbf{Equazione del confine}: $w_1 x_1 + w_2 x_2 + b = 0$
\end{itemize}
\end{frame}
%
%..................................................................
%
\begin{frame}{Dal Confine alla Classificazione}
\textbf{Come usiamo il confine per classificare nuovi punti?}

\vspace{0.5cm}

\begin{block}{Equazione del Confine}
$$w_1 x_1 + w_2 x_2 + b = 0$$
\end{block}

\begin{block}{Regola di Classificazione}
Per un nuovo punto $(x_1, x_2)$:
\begin{itemize}
    \item Se $w_1 x_1 + w_2 x_2 + b > 0$ → \textcolor{blue}{\textbf{Classe A (Spam)}}
    \item Se $w_1 x_1 + w_2 x_2 + b < 0$ → \textcolor{red}{\textbf{Classe B (Non Spam)}}
\end{itemize}
\end{block}

\end{frame}
%
%..................................................................
%
\begin{frame}{Dal Confine alla Classificazione}
\textbf{Come usiamo il confine per classificare nuovi punti?}
\vspace{0.5cm}

\begin{columns}
\begin{column}{0.5\textwidth}
\begin{center}
\begin{tikzpicture}[scale=0.6]
\draw[->] (0,0) -- (4,0) node[right] {$x_1$};
\draw[->] (0,0) -- (0,4) node[above] {$x_2$};

% Confine
\draw[thick, green] (1,0.5) -- (3,3.5);

% Punti di esempio
\fill[blue] (3.5,3) circle (3pt);
\fill[red] (1,1) circle (3pt);

% Nuovo punto
\fill[orange]    (2.5,1.5) circle (4pt);
\node[orange] at (3.0,1.2) {\textbf{?}};

% Etichette delle zone
\node[blue] at (3.5, 4.5) {$w_1x_1+w_2x_2+b>0$};
\node[red]  at (0.5,-1.0) {$w_1x_1+w_2x_2+b<0$};
\end{tikzpicture}
\end{center}
\end{column}
\begin{column}{0.5\textwidth}
\textbf{Esempio Numerico}
\vspace{0.2cm}
Se il confine è:
$$2x_1 + 3x_2 - 5 = 0$$

Per il punto arancione $(2.5, 1.5)$:
$$2(2.5) + 3(1.5) - 5 = 4.5 > 0$$

Quindi: \textcolor{blue}{\textbf{Spam}}!
\end{column}
\end{columns}
\end{frame}
%
%..................................................................
%
\begin{frame}{Torniamo ai Pesi: La Connessione}
\textbf{I pesi $w_1, w_2, b$ definiscono completamente il confine di separazione!}

\vspace{0.5cm}

\begin{columns}
\begin{column}{0.5\textwidth}
\begin{center}
\begin{tikzpicture}[scale=0.7]
\draw[->] (0,0) -- (4,0) node[right] {$x_1$};
\draw[->] (0,0) -- (0,4) node[above] {$x_2$};

% Confine
\draw[thick, green] (0.5,1) -- (3.5,3);

% Vettore normale
%\draw[->, thick, purple] (2,2) -- (2.5,2.8);
\node[purple] at (2.8,0.5) {$(w_1,w_2)$};

% Punti
\fill[blue] (3,3.5) circle (2pt);
\fill[red] (1,0.5) circle (2pt);
\end{tikzpicture}
\end{center}
\end{column}
\begin{column}{0.5\textwidth}
\begin{alertblock}{Collegamento Cruciale}
Il \textbf{training} del modello consiste nel trovare i pesi $(w_1, w_2, b)$ che definiscono il confine ottimale!
\end{alertblock}

\textbf{Processo}:
\begin{enumerate}
    \item Prova diversi confini (diversi pesi)
    \item Misura quanti punti classifica male
    \item Aggiusta i pesi per ridurre gli errori
\end{enumerate}
\end{column}
\end{columns}
\end{frame}
%
%..................................................................
%
\begin{frame}{Sintesi: Una Nuova Prospettiva}
\begin{block}{Abbiamo trasformato il problema!}
\textbf{Da}: "Come classificare email spam?" \\
\textbf{A}: "Come trovare il miglior confine geometrico in uno spazio multidimensionale?"
\end{block}

\vspace{0.5cm}

\textbf{Vantaggi di questa visione geometrica:}
\begin{itemize}
    \item \textbf{Intuizione}: Capiamo cosa fa veramente l'algoritmo
    \item \textbf{Generalizzazione}: Funziona per qualsiasi numero di dimensioni
    \item \textbf{Ottimizzazione}: Possiamo usare metodi geometrici potenti
\end{itemize}

\vspace{0.5cm}

\begin{block}{Concetto Universale}
\textbf{Machine Learning} = Trovare pattern e confini negli spazi multidimensionali!
\end{block}
\end{frame}

%% aggiungere esempio del cancro al seno 


%
%=====================================================================
%
\end{document}
