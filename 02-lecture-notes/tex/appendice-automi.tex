\documentclass{beamer}
%
%=====================================================================
%
\usetheme{CambridgeUS}
%
%=====================================================================
%
\usepackage[utf8]{inputenc}
\usepackage{graphicx}
\usepackage{hyperref}
%
%=====================================================================
%
\title{Storie di Automi Attraverso i Secoli}
\subtitle{Dalle Metamorfosi di Ovidio alla Fantascienza Moderna}
\setbeamercovered{transparent} 
\author{Giovanni Della Lunga\\{\footnotesize giovanni.dellalunga@unibo.it}}
\institute{A lezione di Intelligenza Artificiale} 
\date{Siena - Giugno 2025} 
%
%=====================================================================
%
\begin{document}

% Slide titolo
\begin{frame}
    \titlepage
\end{frame}

% Indice
\begin{frame}{Indice}
    \tableofcontents
\end{frame}
%
%=====================================================================
%
\AtBeginSection[]
{
  \begin{frame}
  \vfill
  \centering
  \begin{beamercolorbox}[sep=8pt,center,shadow=true,rounded=true]{title}  	 	 	\usebeamerfont{title}\insertsectionhead\par%
  \end{beamercolorbox}
  \vfill
  \end{frame}
}
\AtBeginSubsection{\frame{\subsectionpage}}
%__________________________________________________________________________
%
\section{Introduzione}
%
%..................................................................
%
\begin{frame}{Perché studiare gli automi?}
  \begin{itemize}
    \item L'automa incarna il sogno (e l'incubo) dell'uomo di creare vita artificiale.
    \item Attraverso i secoli, gli automi riflettono paure, speranze e fantasie di ogni epoca.
    \item Storia, mito, scienza e immaginazione si intrecciano attorno a queste figure.
    \item Questa presentazione esplora le radici culturali e letterarie degli automi.
  \end{itemize}
\end{frame}
%__________________________________________________________________________
%
\section{Antichità classica}
%
%..................................................................
%
\begin{frame}{Gli automi nell'antichità}
  \begin{itemize}
    \item \textbf{Metamorfosi di Ovidio (I sec. d.C.)}: Pigmalione crea una statua così perfetta da innamorarsene. Afrodite la anima.
    \item \textbf{Talos}: gigante di bronzo costruito da Efesto, difensore di Creta (mitologia greca).
    \item \textbf{Erone di Alessandria (I sec. d.C.)}: ingegnere greco che progettò automi meccanici funzionanti con vapore e contrappesi.
    \item L'automa è al tempo stesso prodigio tecnico e figura mitica.
  \end{itemize}
\end{frame}
%__________________________________________________________________________
%
\section{Medioevo e Rinascimento}
%
%..................................................................
%
\begin{frame}{Automi nel mondo medievale}
  \begin{itemize}
    \item \textbf{Il Golem}: creatura d’argilla animata dalla parola sacra (tradizione ebraica, Praga, XVI sec.).
    \item \textbf{Leggende islamiche e cinesi}: palazzi meccanici, uccelli canori automatici.
    \item \textbf{Alberti, Leonardo da Vinci, Villard de Honnecourt}: studi di dispositivi meccanici e “figure animabili”.
    \item L’automa è al confine tra magia, religione e tecnologia.
  \end{itemize}
\end{frame}
%__________________________________________________________________________
%
\section{Illuminismo e Rivoluzione Industriale}
%
%..................................................................
%
\begin{frame}{L'automa nell'età della ragione}
  \begin{itemize}
    \item XVIII secolo: fioritura degli automi meccanici in Europa (orologiai svizzeri, francesi).
    \item \textbf{Vaucanson}: anatra meccanica che mangiava, digeriva e si muoveva.
    \item \textbf{Jaquet-Droz}: automi in grado di scrivere, disegnare e suonare.
    \item L'automa diventa simbolo della razionalità e della perfezione meccanica.
  \end{itemize}
\end{frame}
%
%..................................................................
%
\begin{frame}{Il Turco Meccanico: l'inganno geniale}
  \begin{itemize}
    \item Creato da \textbf{Wolfgang von Kempelen} nel 1770 per stupire la corte di Maria Teresa d’Austria.
    \item Sembrava un automa capace di giocare a scacchi contro umani e vincere!
    \item Vestito in stile "turco ottomano", era seduto a un tavolo con una scacchiera.
    \item In realtà, un abile giocatore umano era nascosto all’interno della macchina.
  \end{itemize}
\end{frame}
%
%..................................................................
%
\begin{frame}{Il fascino del Turco Meccanico}
  \begin{itemize}
    \item Il dispositivo girò l’Europa e fu persino ricostruito da \textbf{Johann Nepomuk Mälzel} nel XIX secolo.
    \item Tra gli avversari del Turco: Napoleone Bonaparte e Benjamin Franklin.
    \item Il mistero fu svelato solo nel 1857 da Silas Mitchell, il figlio del suo ultimo proprietario.
    \item Il Turco influenzò profondamente l’immaginario dei primi automi e ispirò racconti fantastici.
  \end{itemize}
\end{frame}
%__________________________________________________________________________
%
\section{Romanticismo e Modernità}
%
%..................................................................
%
\begin{frame}{Il lato oscuro dell’automa}
  \begin{itemize}
    \item \textbf{E.T.A. Hoffmann, L'uomo della sabbia} (1816): Olimpia, donna-automa inquietante, anticipa il tema del simulacro.
    \item \textbf{Mary Shelley, Frankenstein} (1818): la creatura artificiale come tragedia dell’hybris umana.
    \item Il Romanticismo esplora il confine tra umano e disumano.
    \item L’automa incarna l’angoscia dell’alienazione e del doppio.
  \end{itemize}
\end{frame}
%__________________________________________________________________________
%
\section{XX e XXI secolo}
%
%..................................................................
%
\begin{frame}{Dagli automi ai robot e androidi}
  \begin{itemize}
    \item \textbf{Karel Čapek, R.U.R.} (1920): nasce la parola \emph{robot}.
    \item \textbf{Isaac Asimov}: le Tre Leggi della Robotica. I robot diventano allegorie morali.
    \item \textbf{Philip K. Dick, Blade Runner}: cosa distingue l'umano dall'automa?
    \item Oggi: robot umanoidi, intelligenze artificiali, deepfake. L’automa si evolve in entità digitale.
  \end{itemize}
\end{frame}
%__________________________________________________________________________
%
\section{Dalla meccanica al pensiero}
%
%..................................................................
%
\begin{frame}{Babbage, Lovelace e le origini del calcolo}
  \begin{itemize}
    \item \textbf{Charles Babbage}: ideatore della macchina analitica (1837), primo modello teorico di calcolatore programmabile.
    \item \textbf{Ada Lovelace}: anticipò il concetto di software, riconoscendo il potenziale creativo delle macchine.
    \item La macchina non solo come esecutrice, ma come strumento di rappresentazione del pensiero.
  \end{itemize}
\end{frame}
%
%..................................................................
%
\begin{frame}{Dai calcolatori elettromeccanici al computer moderno}
  \begin{itemize}
    \item Primi calcolatori del XX secolo: ENIAC, Z3, Colossus.
    \item Utilizzo bellico e scientifico durante la Seconda Guerra Mondiale.
    \item Nascita dell'informatica moderna e crescita esponenziale della capacità di calcolo.
  \end{itemize}
\end{frame}
%
%..................................................................
%
\begin{frame}
\begin{itemize}
\item Ogni sistema binario ha la caratteristica, teorizzata nella metà del XIX secolo dal matematico e logico inglese Boole di poter operare sia con numeri che con simboli logici che la proprietà distintiva di tutto l'ecosistema digitale che ci circonda dai computer a tutti i dispositivi elettronici fino all'AI

\item Lo sviluppo del computer, nella metà del secolo scorso, ha dato il via ad un progresso tecnologico enorme e velocissimo, poiché per la prima volta nella storia l'uomo è stato in grado di costruire una macchina capace di svolgere due attività, calcolo numerico e ragionamento logico, che contraddistinguono l'intelligenza umana. 
\end{itemize}

\end{frame}
%
%..................................................................
%
\begin{frame}{Alan Turing e l’Imitation Game}
  \begin{itemize}
    \item \textbf{Alan Turing} teorizza la "macchina universale" (1936): base teorica del computer.
    \item Con l'\textbf{Imitation Game} (1950) introduce il criterio per distinguere una macchina intelligente da un umano.
    \item Nasce l’idea di test dell’intelligenza artificiale.
  \end{itemize}
\end{frame}
%
%..................................................................
%
\begin{frame}{L’intelligenza dentro la macchina}
  \begin{itemize}
    \item L’intelligenza artificiale si sviluppa in parallelo al progresso dell’hardware e del software.
    \item L’imitazione dell’intelligenza non è più solo letteraria o fantastica, ma ingegneristica.
    \item Oggi, l’AI è il naturale erede del sogno antico dell’automa: pensare, apprendere, creare.
  \end{itemize}
\end{frame}
%__________________________________________________________________________
%
\section{Cinema e immaginario dell’automa}
%
%..................................................................
%
\begin{frame}{2001: Odissea nello Spazio (1968)}
  \begin{itemize}
    \item Regia di Stanley Kubrick, soggetto di Arthur C. Clarke.
    \item Il supercomputer HAL 9000 rappresenta l’intelligenza artificiale logica, ma imprevedibile e pericolosa.
    \item L'automa non è solo macchina: ha emozioni, intenzioni, desideri. Un personaggio inquietante.
    \item Il conflitto uomo-macchina si fa esistenziale e metafisico.
  \end{itemize}
\end{frame}
%
%..................................................................
%
\begin{frame}{Terminator (1984-2003)}
  \begin{itemize}
    \item Creato da James Cameron: un cyborg assassino inviato dal futuro per eliminare l’umanità.
    \item Skynet, un’IA bellica, prende coscienza e scatena l’apocalisse.
    \item Il robot non più servo, ma minaccia autonoma e inarrestabile.
    \item Il confine tra umano e macchina si fonde con il concetto di destino tecnologico.
  \end{itemize}
\end{frame}
%
%..................................................................
%
\begin{frame}{Matrix (1999-2021)}
  \begin{itemize}
    \item Saga dei Wachowski: un futuro distopico dove l’umanità vive in una realtà simulata creata dalle macchine.
    \item Gli automi hanno vinto: l’intelligenza artificiale domina e alimenta la propria esistenza con l’energia umana.
    \item L’eroe, Neo, è chiamato a liberare l’umanità: l’automa è simbolo di controllo e illusione.
    \item Un’apoteosi filosofica sul rapporto tra realtà, percezione e coscienza artificiale.
  \end{itemize}
\end{frame}
%__________________________________________________________________________
%
\section{Conclusione}
%
%..................................................................
%
\begin{frame}{Un archetipo senza tempo}
  \begin{itemize}
    \item Gli automi sono specchi dell’anima umana: mostrano ciò che desideriamo e temiamo.
    \item Il confine tra artificiale e naturale è sempre più sfumato.
    \item Mito, tecnica e letteratura continuano a dialogare.
    \item L’automa è un simbolo, un monito, una speranza: \textit{creeremo una nuova vita o solo una copia?}
  \end{itemize}
\end{frame}
%__________________________________________________________________________
%
\begin{frame}[allowframebreaks]{Riferimenti e fonti}
  \small
  \begin{itemize}
    \item Riskin, Jessica. \textit{The Restless Clock}. University of Chicago Press, 2016.
    \item Asimov, Isaac. \textit{Io, Robot}. 1950.
    \item Čapek, Karel. \textit{R.U.R. (Rossum's Universal Robots)}. 1920.
    \item Ovidio, \textit{Metamorfosi}.
    \item Racconti ebraici sul Golem di Praga.
    \item Standage, Tom. \textit{The Turk: The Life and Times of the Famous Eighteenth-Century Chess-Playing Machine}. Penguin, 2003.
    \item Babbage, Charles. \textit{Passages from the Life of a Philosopher}. 1864.
    \item Turing, Alan M. \textit{Computing Machinery and Intelligence}. Mind, 1950.
    \item Lovelace, Ada. \textit{Notes on the Analytical Engine}, 1843.
  \end{itemize}
\end{frame}
%
%=====================================================================
%
\end{document}
%
%=====================================================================
%
