\documentclass[aspectratio=169]{beamer}
%
%=====================================================================
%
\usepackage[utf8]{inputenc}
\usepackage[italian]{babel}
\usepackage{amsmath}
\usepackage{graphicx}
\usepackage{tikz}
\usepackage{xcolor}
%
%=====================================================================
%
% Tema e colori
\usetheme{Madrid}
\usecolortheme{seahorse}
%
%=====================================================================
%
% Definizione colori personalizzati
\definecolor{aiblue}{RGB}{30,144,255}
\definecolor{mlgreen}{RGB}{46,139,87}
\definecolor{datacolor}{RGB}{255,140,0}
%
%=====================================================================
%
\title{Dimensione Operativa \\ 1. AI Chatbot ed altri Tools}
\subtitle{}
\setbeamercovered{transparent} 
\author{Giovanni Della Lunga\\{\footnotesize giovanni.dellalunga@unibo.it}}
\institute{A lezione di Intelligenza Artificiale} 
\date{Siena - Giugno 2025} 
%
%=====================================================================
%
\begin{document}

% Slide titolo
\begin{frame}
    \titlepage
\end{frame}

% Indice
\begin{frame}{Indice}
    \tableofcontents
\end{frame}
%
%=====================================================================
%
\AtBeginSection[]
{
  \begin{frame}
  \vfill
  \centering
  \begin{beamercolorbox}[sep=8pt,center,shadow=true,rounded=true]{title}  	 	 	\usebeamerfont{title}\insertsectionhead\par%
  \end{beamercolorbox}
  \vfill
  \end{frame}
}
\AtBeginSubsection{\frame{\subsectionpage}}
%
%=====================================================================
%
\begin{frame}
\frametitle{AI chatbot}
\begin{itemize}
    \item Generalmente l’interfaccia utente include, come elemento fondamentale, una casella di testo nella quale l’utente inserisce la richiesta (detta prompt) e uno spazio nel quale il sistema produce la risposta.
    \item Nella maggior parte dei casi il servizio viene offerto con una modalità gratuita oppure, con funzionalità più avanzate, a pagamento.
\end{itemize}
\end{frame}
%__________________________________________________________________________
%
\section{ChatGPT}
%
%..................................................................
%
\begin{frame}
\frametitle{ChatGPT}
\begin{itemize}
    \item ChatGPT (https://chatgpt.com), sviluppato da OpenAI, è senza dubbio uno dei chatbot più conosciuti e utilizzati.
    \item La sua interfaccia è semplice: una casella di testo in cui inserire domande o richieste, e nel box principale l’intelligenza artificiale risponde, tenendo conto del contesto.
    \item La sua vera forza è la versatilità: ChatGPT eccelle in contesti creativi, che si tratti di scrittura, risoluzione di problemi complessi o conversazioni casuali.
    \item Nella versione a pagamento è particolarmente completo, con la capacità, ad esempio, di generare immagini, oltre che testo.
    \item Può essere particolarmente utile anche nel campo della programmazione, offrendo suggerimenti e spiegazioni a chi cerca assistenza con codice e algoritmi.
\end{itemize}
\end{frame}
%
%..................................................................
%
\begin{frame}
\frametitle{Descrizione dell'interfaccia}
L’ interfaccia è essenziale e intuitiva, con risposte rapide e facilità di interazione. Supporta conversazioni fluide e continuative, mantenendo il contesto durante le conversazioni.\\
\vspace{0.5cm}
\begin{itemize}
    \item Panoramica della schermata principale
    \item Navigazione tra le funzioni disponibili
    \item Utilizzo della barra di ricerca per le query
\end{itemize}
\end{frame}
%
%..................................................................
%
\begin{frame}{ChatGPT}
\begin{center}
\includegraphics[scale=.35]{../05-pictures/dimensione-operativa-1_pic_0.png} 
\end{center}
\end{frame}
%
%..................................................................
%
\begin{frame}
\frametitle{Identificazione dell'utente}
\begin{itemize}
    \item Creazione di un account per l'accesso
    \item Importanza della registrazione
    \item Gestione delle impostazioni dell'account
\end{itemize}
\end{frame}
%
%..................................................................
%
\begin{frame}{ChatGPT}
\begin{center}
\includegraphics[scale=.35]{../05-pictures/dimensione-operativa-1_pic_1.png} 
\end{center}
\end{frame}
%
%..................................................................
%
\begin{frame}{ChatGPT}
\begin{center}
\includegraphics[scale=.35]{../05-pictures/dimensione-operativa-1_pic_2.png} 
\end{center}
\end{frame}
%
%..................................................................
%
\begin{frame}{ChatGPT}
\begin{center}
\includegraphics[scale=.35]{../05-pictures/dimensione-operativa-1_pic_3.png} 
\end{center}
\end{frame}
%
%..................................................................
%
\begin{frame}{ChatGPT}
\begin{center}
\includegraphics[scale=.35]{../05-pictures/dimensione-operativa-1_pic_4.png} 
\end{center}
\end{frame}
%
%..................................................................
%
\begin{frame}{ChatGPT}
\begin{center}
\includegraphics[scale=.35]{../05-pictures/dimensione-operativa-1_pic_5.png} 
\end{center}
\end{frame}
%
%..................................................................
%
\begin{frame}{ChatGPT}
\begin{center}
\includegraphics[scale=.35]{../05-pictures/dimensione-operativa-1_pic_6.png} 
\end{center}
\end{frame}
%
%..................................................................
%
\begin{frame}{ChatGPT}
\begin{center}
\includegraphics[scale=.35]{../05-pictures/dimensione-operativa-1_pic_7.png} 
\end{center}
\end{frame}
%
%..................................................................
%
\begin{frame}{ChatGPT}
\begin{center}
\includegraphics[scale=.35]{../05-pictures/dimensione-operativa-1_pic_8.png} 
\end{center}
\end{frame}
%
%..................................................................
%
\begin{frame}{ChatGPT}
\begin{center}
\includegraphics[scale=.35]{../05-pictures/dimensione-operativa-1_pic_9.png} 
\end{center}
\end{frame}
%
%..................................................................
%
\begin{frame}{ChatGPT}
\begin{center}
\includegraphics[scale=.35]{../05-pictures/dimensione-operativa-1_pic_10.png} 
\end{center}
\end{frame}
%
%..................................................................
%
\begin{frame}
\frametitle{ChatGPT}
\begin{itemize}
    \item A partire dal modello o1, OpenAI promette capacità di ragionamento avanzato su problemi complessi al fine di affrontare con maggiore efficacia problemi di logica articolata, un aspetto fondamentale per compiti di analisi e problem-solving in ambito educativo.
\end{itemize}
\end{frame}
%
%..................................................................
%
\begin{frame}{Cosa li distingue dai modelli precedenti}
  \begin{itemize}
    \item Nessuna necessità di convertire input visivi o vocali in testo intermedio: l'elaborazione è \textbf{end-to-end}.
    \item Capacità avanzate di \textbf{ragionamento visuale e logico combinato}.
    \item \textbf{Risposta più umana}: intonazione, pause e comprensione dei turni conversazionali.
    \item Ideale per compiti complessi dove serve \textbf{integrazione multimodale} (es. medicina, istruzione, sicurezza).
  \end{itemize}
\end{frame}
%
%..................................................................
%
\begin{frame}{Un esempio concreto di utilizzo...}
  \begin{itemize}
    \item Supponiamo che un docente desideri realizzare della documentazione a supporto di alcuni problemi di matematica ...
    \item Vediamo come procedere a partire da una semplice foto dell'esercizio 
  \end{itemize}
\end{frame}
%
%..................................................................
%
\begin{frame}
\frametitle{ChatGPT}
\begin{itemize}
    \item ChatGPT ha anche introdotto i canvas, uno strumento integrato progettato per facilitare la creazione, l’editing e la gestione di contenuti complessi, come documenti o codice.
    \item Questo strumento abilità una modalità di lavoro nella quale il contenuto da creare si trova in un ambiente dedicato, separato dalla conversazione principale e offre una maggiore flessibilità per progetti articolati.
\end{itemize}
\end{frame}
%__________________________________________________________________________
%
\section{Claude}
%
%..................................................................
%
\begin{frame}{Chi è Anthropic}
\begin{itemize}
    \item \textbf{Fondata nel 2021} da ex-membri di OpenAI, inclusi Dario Amodei e Daniela Amodei
    \item \textbf{Mission}: Sviluppare sistemi di IA sicuri, benefici e comprensibili
    \item \textbf{Focus principale}: Ricerca sulla sicurezza dell'IA e allineamento
    \item \textbf{Finanziamenti}: Oltre 750 milioni di dollari (Google, Spark Capital)
    \item \textbf{Approccio}: Ricerca fondamentale combinata con sviluppo di prodotti commerciali
\end{itemize}

\vspace{0.5cm}
\textbf{Differenza chiave}: Priorità esplicita sulla sicurezza e interpretabilità rispetto alla pura performance
\end{frame}
%
%..................................................................
%
\begin{frame}{Filosofia: Constitutional AI}
\begin{itemize}
    \item \textbf{Constitutional AI (CAI)}: Metodologia sviluppata da Anthropic per l'allineamento
    \item \textbf{Principio base}: Il modello apprende un insieme di principi morali e etici
    \item \textbf{Due fasi principali}:
    \begin{enumerate}
        \item \textbf{Supervised Learning}: Training su conversazioni che seguono principi costituzionali
        \item \textbf{Reinforcement Learning}: Ottimizzazione basata su feedback che rispetta la "costituzione"
    \end{enumerate}
    \item \textbf{Vantaggi}:
    \begin{itemize}
        \item Riduzione di output dannosi o pregiudizievoli
        \item Maggiore trasparenza nei processi decisionali
        \item Scalabilità dell'allineamento
    \end{itemize}
\end{itemize}
\end{frame}
%
%..................................................................
%
\begin{frame}{Evoluzione del Modello Claude}
\begin{table}[h]
\centering
\small
\begin{tabular}{|l|l|l|l|}
\hline
\textbf{Versione} & \textbf{Data Rilascio} & \textbf{Caratteristiche Principali} & \textbf{Context Window} \\
\hline
Claude 1.0 & Marzo 2022 & Primo modello con Constitutional AI & 9K token \\
\hline
Claude 1.3 & Maggio 2022 & Miglioramenti conversazionali & 9K token \\
\hline
Claude 2.0 & Luglio 2023 & Context window esteso, coding & 100K token \\
\hline
Claude 2.1 & Novembre 2023 & Riduzione allucinazioni, tool use & 200K token \\
\hline
Claude 3 Haiku & Marzo 2024 & Modello veloce ed economico & 200K token \\
\hline
Claude 3 Sonnet & Marzo 2024 & Bilanciamento performance/costo & 200K token \\
\hline
Claude 3 Opus & Marzo 2024 & Modello più potente della famiglia & 200K token \\
\hline
\end{tabular}
\end{table}

\textbf{Trend evolutivo}: Da modello singolo a famiglia di modelli specializzati per diversi use case
\end{frame}
%
%..................................................................
%
\subsection{Architettura e Funzionamento}
%
%..................................................................
%
\begin{frame}{Architettura Base}
\begin{itemize}
    \item \textbf{Architettura}: Transformer decoder-only (simile a GPT)
    \item \textbf{Parametri}: 
    \begin{itemize}
        \item Claude 3 Haiku: $\sim$20B parametri (stimato)
        \item Claude 3 Sonnet: $\sim$70B parametri (stimato)
        \item Claude 3 Opus: $\sim$175B parametri (stimato)
    \end{itemize}
    \item \textbf{Innovazioni architetturali}:
    \begin{itemize}
        \item Attention mechanisms ottimizzati per context window lunghi
        \item Layer specializzati per Constitutional AI
        \item Architettura modulare per diverse capacità (testo, codice, reasoning)
    \end{itemize}
    \item \textbf{Training Infrastructure}: Cluster di GPU/TPU custom, training distribuito
\end{itemize}
\end{frame}
%
%..................................................................
%
\begin{frame}{Dati di Addestramento}
\begin{itemize}
    \item \textbf{Dataset}: Combinazione di fonti pubbliche e curate (dettagli limitati per proprietà intellettuale)
    \item \textbf{Fonti principali}:
    \begin{itemize}
        \item Web crawling filtrato e curato
        \item Libri e letteratura
        \item Articoli scientifici e accademici
        \item Codice sorgente da repository pubblici
        \item Dataset conversazionali sintetici
    \end{itemize}
    \item \textbf{Filtri di qualità}:
    \begin{itemize}
        \item Rimozione contenuti dannosi, bias, disinformazione
        \item Deduplicazione avanzata
        \item Controlli di qualità linguistica
    \end{itemize}
    \item \textbf{Cutoff temporale}: Variabile per versione (Claude 3: inizio 2024)
\end{itemize}
\end{frame}
%
%..................................................................
%
\begin{frame}{Constitutional AI: Implementazione Tecnica}
\begin{enumerate}
    \item \textbf{Phase 1 - Supervised Fine-tuning}:
    \begin{itemize}
        \item Training su conversazioni che rispettano principi costituzionali
        \item Critiche e revisioni guidate dai principi
        \item Self-correction loops durante il training
    \end{itemize}
    
    \item \textbf{Phase 2 - Reinforcement Learning from AI Feedback (RLAIF)}:
    \begin{itemize}
        \item Il modello valuta le proprie risposte secondo la costituzione
        \item Training RL basato su questo auto-feedback
        \item Riduzione dipendenza da feedback umano costoso
    \end{itemize}
\end{enumerate}

\vspace{0.3cm}
\textbf{Risultato}: Modello intrinsecamente allineato che bilancia helpfulness, harmlessness e honesty
\end{frame}
%
%..................................................................
%
\begin{frame}{Capacità Principali}
\begin{itemize}
    \item \textbf{Comprensione del linguaggio naturale}:
    \begin{itemize}
        \item Analisi semantica avanzata
        \item Comprensione di contesti complessi e sfumature
        \item Multilingual capabilities (100+ lingue)
    \end{itemize}
    
    \item \textbf{Ragionamento logico}:
    \begin{itemize}
        \item Mathematical reasoning e problem solving
        \item Causal reasoning e inferenze complesse
        \item Chain-of-thought reasoning esplicito
    \end{itemize}
    
    \item \textbf{Programmazione}:
    \begin{itemize}
        \item Code generation in 20+ linguaggi
        \item Debugging e code review
        \item Architectural design e best practices
    \end{itemize}
    
    \item \textbf{Sintesi e analisi}:
    \begin{itemize}
        \item Document summarization per testi lunghi
        \item Data analysis e pattern recognition
        \item Multi-document synthesis
    \end{itemize}
\end{itemize}
\end{frame}
%
%..................................................................
%
\subsection{Punti di Forza e Vantaggi}
%
%..................................................................
%
\begin{frame}{Performance nei Benchmark}
\begin{table}[h]
\centering
\small
\begin{tabular}{|l|c|c|c|c|}
\hline
\textbf{Benchmark} & \textbf{Claude 3 Opus} & \textbf{GPT-4} & \textbf{Gemini Ultra} & \textbf{LLaMA 2 70B} \\
\hline
MMLU & 86.8\% & 86.4\% & 83.7\% & 68.9\% \\
\hline
HumanEval (Coding) & 84.9\% & 67.0\% & 74.4\% & 29.9\% \\
\hline
GSM8K (Math) & 95.0\% & 92.0\% & 94.4\% & 56.8\% \\
\hline
HellaSwag & 95.4\% & 95.3\% & 87.8\% & 85.3\% \\
\hline
ARC-Challenge & 96.4\% & 96.3\% & 94.1\% & 85.8\% \\
\hline
\end{tabular}
\end{table}

\textbf{Punti salienti}:
\begin{itemize}
    \item Leadership in coding tasks (HumanEval)
    \item Performance competitive o superiore in reasoning
    \item Consistenza across different domains
\end{itemize}
\end{frame}
%
%..................................................................
%
\begin{frame}{Context Window: Vantaggio Competitivo}
\begin{itemize}
    \item \textbf{200K token context}: Tra i più lunghi disponibili commercialmente
    \item \textbf{Equivalente}: $\sim$150,000 parole o $\sim$500 pagine di testo
    \item \textbf{Vantaggi pratici}:
    \begin{itemize}
        \item Analisi di documenti interi (contratti, report, codebases)
        \item Conversazioni lunghe senza perdita di contesto
        \item Multi-document analysis simultanea
        \item Maintained coherence over extended interactions
    \end{itemize}
    \item \textbf{Implementazione tecnica}:
    \begin{itemize}
        \item Attention patterns ottimizzati per scaling
        \item Memory management efficiente
        \item Retrieval mechanisms integrati
    \end{itemize}
\end{itemize}

\textbf{Use case}: Analisi legale, ricerca accademica, software engineering
\end{frame}
%
%..................................................................
%
\begin{frame}{Riduzione delle Allucinazioni}
\begin{itemize}
    \item \textbf{Problema delle allucinazioni}: Generazione di informazioni false ma plausibili
    \item \textbf{Approccio di Anthropic}:
    \begin{itemize}
        \item Constitutional AI training specifico per accuracy
        \item "I don't know" responses quando appropriato
        \item Source citation e uncertainty quantification
        \item Multi-step verification processes
    \end{itemize}
    \item \textbf{Risultati misurabili}:
    \begin{itemize}
        \item Riduzione 40-60\% di false claims rispetto a baseline
        \item Maggiore accuracy in factual questions
        \item Improved calibration (confidence matching accuracy)
    \end{itemize}
    \item \textbf{Meccanismi tecnici}:
    \begin{itemize}
        \item Adversarial training contro hallucinations
        \item Factual consistency checks durante generation
        \item Uncertainty-aware decoding strategies
    \end{itemize}
\end{itemize}
\end{frame}
%
%..................................................................
%
\begin{frame}{Affidabilità in Contesti Professionali}
\begin{itemize}
    \item \textbf{Enterprise adoption}: Integrazione in workflow aziendali critici
    \item \textbf{Caratteristiche per uso professionale}:
    \begin{itemize}
        \item Consistent behavior e predictable outputs
        \item Compliance con standard di sicurezza (SOC 2, GDPR)
        \item API stability e SLA garantiti
        \item Fine-tuning capabilities per domini specifici
    \end{itemize}
    \item \textbf{Settori di applicazione}:
    \begin{itemize}
        \item Legal: Contract analysis, legal research
        \item Healthcare: Medical literature review (non-diagnostic)
        \item Finance: Risk analysis, regulatory compliance
        \item Education: Curriculum development, assessment
        \item Consulting: Report generation, data analysis
    \end{itemize}
    \item \textbf{Risk management}: Built-in guardrails, audit trails, human oversight integration
\end{itemize}
\end{frame}
%
%..................................................................
%
\subsection{Confronto con Altri LLM}
%
%..................................................................
%
\begin{frame}{Claude vs ChatGPT (OpenAI)}
\begin{table}[h]
\centering
\footnotesize
\begin{tabular}{|p{3cm}|p{4cm}|p{4cm}|}
\hline
\textbf{Aspetto} & \textbf{Claude} & \textbf{ChatGPT} \\
\hline
\textbf{Filosofia} & Constitutional AI, safety-first & Performance-focused, iterative alignment \\
\hline
\textbf{Context Window} & 200K token & 32K token (GPT-4 Turbo) \\
\hline
\textbf{Coding} & Superiore in benchmark & Molto buono, ma inferiore \\
\hline
\textbf{Allucinazioni} & Più conservative, meno frequent & Più creative, più prone \\
\hline
\textbf{Multimodalità} & Testo + immagini (input) & Testo + immagini + audio \\
\hline
\textbf{Ecosistema} & API-focused & ChatGPT + plugins + GPT Store \\
\hline
\textbf{Pricing} & Competitivo, pay-per-use & Subscription + API \\
\hline
\end{tabular}
\end{table}

\textbf{Conclusione}: Claude per accuracy e safety, ChatGPT per ecosistema e features
\end{frame}
%
%..................................................................
%
\begin{frame}{Claude vs Gemini (Google)}
\begin{table}[h]
\centering
\footnotesize
\begin{tabular}{|p{3cm}|p{4cm}|p{4cm}|}
\hline
\textbf{Aspetto} & \textbf{Claude} & \textbf{Gemini} \\
\hline
\textbf{Architettura} & Transformer decoder-only & Multimodal Transformer nativo \\
\hline
\textbf{Multimodalità} & Aggiunta post-training & Nativa dall'inizio \\
\hline
\textbf{Context Window} & 200K token & 1M token (Gemini Pro) \\
\hline
\textbf{Integrazione} & Third-party focused & Google ecosystem \\
\hline
\textbf{Performance} & Leadership in coding & Competitive across domains \\
\hline
\textbf{Reasoning} & Chain-of-thought esplicito & Implicit reasoning + tools \\
\hline
\textbf{Privacy} & Anthropic data policies & Google data integration \\
\hline
\end{tabular}
\end{table}

\textbf{Trade-off}: Claude per indipendenza e coding, Gemini per multimodalità e scale
\end{frame}
%
%..................................................................
%
\begin{frame}{Claude vs Open Source (LLaMA, Mistral)}
\begin{itemize}
    \item \textbf{LLaMA 2/3 (Meta)}:
    \begin{itemize}
        \item \textbf{Vantaggi}: Open source, customizable, no vendor lock-in
        \item \textbf{Svantaggi}: Performance inferiore, requires infrastructure
        \item \textbf{Claude advantage}: Superior performance, managed service
    \end{itemize}
    
    \item \textbf{Mistral}:
    \begin{itemize}
        \item \textbf{Vantaggi}: European focus, competitive performance/cost
        \item \textbf{Svantaggi}: Smaller context, less mature ecosystem
        \item \textbf{Claude advantage}: Longer context, Constitutional AI
    \end{itemize}
    
    \item \textbf{Considerazioni strategiche}:
    \begin{itemize}
        \item \textbf{Open source}: Control, cost, compliance
        \item \textbf{Claude}: Performance, safety, managed service
        \item \textbf{Hybrid approaches}: Claude for critical tasks, open source for bulk
    \end{itemize}
\end{itemize}
\end{frame}
%
%..................................................................
%
\begin{frame}{Differenze Filosofiche nell'Allineamento}
\begin{itemize}
    \item \textbf{Anthropic (Constitutional AI)}:
    \begin{itemize}
        \item Principi espliciti e trasparenti
        \item Self-correction e auto-miglioramento
        \item Focus su interpretabilità
    \end{itemize}
    
    \item \textbf{OpenAI (RLHF)}:
    \begin{itemize}
        \item Human feedback diretto
        \item Iterative improvement
        \item Performance-safety balance
    \end{itemize}
    
    \item \textbf{Google (Constitutional + Factuality)}:
    \begin{itemize}
        \item Integration con knowledge graphs
        \item Real-time fact checking
        \item Multi-step verification
    \end{itemize}
    
    \item \textbf{Meta (Community-driven)}:
    \begin{itemize}
        \item Open development process
        \item Community feedback integration
        \item Distributed safety research
    \end{itemize}
\end{itemize}
\end{frame}
%
%..................................................................
%
\begin{frame}{Conclusioni}
\begin{itemize}
    \item \textbf{Claude rappresenta un approccio distintivo} nell'evoluzione dei Large Language Models
    \item \textbf{Punti di forza chiave}:
    \begin{itemize}
        \item Constitutional AI per safety e reliability
        \item Context window esteso per applicazioni complesse
        \item Performance superiore in coding e reasoning
        \item Focus su interpretabilità e trasparenza
    \end{itemize}
    \item \textbf{Posizionamento strategico}: Premium model per use case professionali critici
    \item \textbf{Contributo all'ecosistema}: Leadership nella ricerca su AI safety e alignment
    \item \textbf{Prospettive}: Continued innovation in constitutional approaches e scalable oversight
\end{itemize}

\vspace{0.5cm}
\textbf{Takeaway}: Claude dimostra che performance e safety non sono mutually exclusive, ma possono essere co-optimized attraverso principled approaches
\end{frame}
%__________________________________________________________________________
%
\section{Perplexity}
%
%..................................................................
%
\begin{frame}{Chi è Perplexity AI}
\begin{itemize}
    \item \textbf{Fondata nel 2022} da Aravind Srinivas, Denis Yarats, Johnny Ho e Andy Konwinski
    \item \textbf{CEO Aravind Srinivas}: Ex-ricercatore OpenAI, background in AI e NLP
    \item \textbf{Lancio pubblico}: Dicembre 2022 con motore di ricerca conversazionale
    \item \textbf{Valutazione}: 18 miliardi di dollari (2025).
    \item \textbf{Crescita}: 10 milioni di utenti attivi mensili nel 2023
    \item \textbf{Mission}: "Il modo più veloce per ottenere risposte a qualsiasi domanda"
\end{itemize}

\vspace{0.5cm}
\textbf{Posizionamento}: Competitore diretto di Google Search con approccio AI-first
\end{frame}
%
%..................................................................
%
\begin{frame}{Filosofia: Conversational Search}
\begin{itemize}
    \item \textbf{Paradigma tradizionale}: Search → Links → Manual synthesis
    \item \textbf{Paradigma Perplexity}: Question → Direct answer + Sources
    \item \textbf{Principi fondamentali}:
    \begin{enumerate}
    \normalsize
        \item \textbf{Accuratezza}: Ogni risposta citata e verificabile
        \item \textbf{Freschezza}: Informazioni sempre aggiornate via real-time search
        \item \textbf{Trasparenza}: Sources visibili e accessibili
        \item \textbf{Conversazionalità}: Follow-up naturali e contestuali
    \end{enumerate}
    \item \textbf{Differenza chiave}: Non sostituisce il pensiero critico, ma accelera la research
    \item \textbf{Target}: Ricercatori, studenti, professionisti, curiosi
\end{itemize}

\vspace{0.3cm}
\textbf{Obiettivo}: Democratizzare l'accesso a informazioni accurate e tempestive
\end{frame}
%
%..................................................................
%
\begin{frame}{Architettura}
\begin{itemize}
\item Perplexity non ha un modello linguistico (o LLM) proprietario e questo rappresenta uno svantaggio competitivo rispetto ad aziende come Google che ha sviluppato Gemini o Meta con Llama.
\item La piattaforma si basa su un modello linguistico di grandi dimensioni (LLM) che combina le capacità di GPT-4 di OpenAI, integrandoli con la capacità di accedere in tempo reale alle informazioni presenti sul web. 
\item Per alcune specifiche attività di ricerca viene utilizzato anche Claude di Anthropic.
\end{itemize}
\end{frame}
%__________________________________________________________________________
%
\section{Gemini}
%
%..................................................................
%
\begin{frame}
\frametitle{Introduzione a Gemini}
\begin{itemize}
    \item \textbf{Sviluppatore}: Google DeepMind (precedentemente Google AI)
    \item \textbf{Lancio}: Dicembre 2023 (Gemini 1.0), evoluzione da Bard
    \item \textbf{Obiettivi del progetto}:
    \begin{itemize}
        \item Superare le limitazioni di LaMDA e PaLM
        \item Creare un modello nativo multimodale
        \item Competere direttamente con GPT-4 e modelli all'avanguardia
    \end{itemize}
    \item \textbf{Evoluzione}: Bard $\rightarrow$ LaMDA/PaLM $\rightarrow$ Gemini
    \item \textbf{Versioni}: Gemini Nano, Pro, Ultra (1.0) e Gemini 1.5 Pro/Flash
\end{itemize}
\end{frame}
%
%..................................................................
%
\begin{frame}
\frametitle{Architettura e Funzionamento}
\begin{itemize}
    \item \textbf{Architettura di base}: Transformer evoluto con architettura decoder-only
    \item \textbf{Multimodalità nativa}:
    \begin{itemize}
        \item Testo, codice, immagini, video, audio
        \item Elaborazione congiunta invece di fusione tardiva
    \end{itemize}
    \item \textbf{Scale del modello}:
    \begin{itemize}
        \item \textbf{Nano}: 1.8B/3.25B parametri (dispositivi mobili)
        \item \textbf{Pro}: Bilanciamento performance/efficienza
        \item \textbf{Ultra}: Modello più grande per compiti complessi
    \end{itemize}
    \item \textbf{Addestramento}: Mixture of Experts (MoE) in Gemini 1.5
    \item \textbf{Finestra di contesto}: Fino a 1 milione di token (Gemini 1.5)
\end{itemize}
\end{frame}
%
%..................................................................
%
\begin{frame}
\frametitle{Vantaggi e Caratteristiche Distintive}
\begin{itemize}
    \item \textbf{Performance su benchmark}:
    \begin{itemize}
        \item MMLU: 90.0\% (Ultra), superando GPT-4
        \item MATH: 94.4\% con chain-of-thought
        \item GSM8K: 94.4\% in matematica elementare
        \item MMMU: Eccellente su ragionamento multimodale
    \end{itemize}
    \item \textbf{Lunga finestra di contesto}:
    \begin{itemize}
        \item 1M token = $\sim$700,000 parole
        \item Analisi di documenti completi, video lunghi
    \end{itemize}
    \item \textbf{Efficienza}:
    \begin{itemize}
        \item Generazione di codice ad alta qualità
        \item Ragionamento step-by-step migliorato
        \item Velocità di inferenza ottimizzata
    \end{itemize}
\end{itemize}
\end{frame}
%
%..................................................................
%
\begin{frame}
\frametitle{Sicurezza e Allineamento}
\begin{itemize}
    \item \textbf{Strategie di sicurezza}:
    \begin{itemize}
        \item Constitutional AI e RLHF (Reinforcement Learning from Human Feedback)
        \item Red teaming estensivo con team specializzati
        \item Filtering di contenuti dannosi durante addestramento
    \end{itemize}
    \item \textbf{Riduzione delle allucinazioni}:
    \begin{itemize}
        \item Grounding con fonti verificate
        \item Calibrazione dell'incertezza
        \item Fact-checking integrato
    \end{itemize}
    \item \textbf{Testing responsabile}:
    \begin{itemize}
        \item Collaborazione con ricercatori esterni
        \item Valutazione su bias e fairness
        \item Monitoraggio continuo post-deployment
    \end{itemize}
\end{itemize}
\end{frame}
%
%..................................................................
%
\begin{frame}
\frametitle{Confronto Dettagliato - Architettura}
\begin{itemize}
    \item \textbf{Gemini vs GPT-4}:
    \begin{itemize}
        \item Gemini: Multimodalità nativa, MoE in 1.5
        \item GPT-4: Architettura più tradizionale, multimodalità aggiunta
    \end{itemize}
    \item \textbf{Gemini vs Claude}:
    \begin{itemize}
        \item Gemini: Finestra più lunga (1M vs 200K)
        \item Claude: Forte focus su sicurezza e Constitutional AI
    \end{itemize}
    \item \textbf{Gemini vs LLaMA/Mistral}:
    \begin{itemize}
        \item Gemini: Modello closed, risorse computazionali massive
        \item LLaMA/Mistral: Open source, efficienza, personalizzazione
    \end{itemize}
    \item \textbf{Differenze chiave}: Contesto esteso, multimodalità nativa, integrazione Google
\end{itemize}
\end{frame}
%
%..................................................................
%
\begin{frame}
\frametitle{Confronto Performance - Codice e Ragionamento}
\begin{itemize}
    \item \textbf{Generazione di codice (HumanEval)}:
    \begin{itemize}
        \item Gemini Ultra: 74.4\%
        \item GPT-4: 67.0\%
        \item Claude 3 Opus: 84.9\%
    \end{itemize}
    \item \textbf{Ragionamento matematico (GSM8K)}:
    \begin{itemize}
        \item Gemini Ultra: 94.4\%
        \item GPT-4: 92.0\%
        \item Claude 3 Opus: 95.0\%
    \end{itemize}
    \item \textbf{Ragionamento multimodale}:
    \begin{itemize}
        \item Gemini eccelle nell'analisi video/immagini
        \item GPT-4V competitivo ma limitazioni su video
        \item Claude 3 forte su testo + immagini
    \end{itemize}
\end{itemize}
\end{frame}
%
%..................................................................
%
\begin{frame}
\frametitle{Casi d'Uso e Applicazioni Reali}
\begin{itemize}
    \item \textbf{Integrazione prodotti Google}:
    \begin{itemize}
        \item Gmail: Smart compose e risposta intelligente
        \item Google Docs: Scrittura assistita e summarization
        \item Android: Assistente vocale migliorato
        \item YouTube: Analisi e descrizione automatica video
    \end{itemize}
    \item \textbf{Applicazioni enterprise}:
    \begin{itemize}
        \item Gemini API per sviluppatori
        \item Google Cloud AI Platform
        \item Vertex AI integration
    \end{itemize}
    \item \textbf{Strumenti developer}:
    \begin{itemize}
        \item Code completion in IDE
        \item Debugging assistito
        \item Documentazione automatica
    \end{itemize}
\end{itemize}
\end{frame}
%
%..................................................................
%
\begin{frame}
\frametitle{Applicazioni Multimodali Avanzate}
\begin{itemize}
    \item \textbf{Analisi video}:
    \begin{itemize}
        \item Comprensione di contenuti video lunghi (fino a 1 ora)
        \item Estrazione di informazioni da filmati educativi
        \item Analisi di codice in screencast
    \end{itemize}
    \item \textbf{Elaborazione documenti}:
    \begin{itemize}
        \item Analisi di PDF con tabelle e grafici
        \item Estrazione dati da documenti scansionati
        \item Summarization di report tecnici complessi
    \end{itemize}
    \item \textbf{Ricerca scientifica}:
    \begin{itemize}
        \item Analisi di paper con formule matematiche
        \item Interpretazione di grafici e diagrammi
        \item Coding per esperimenti e simulazioni
    \end{itemize}
\end{itemize}
\end{frame}
%
%..................................................................
%
\begin{frame}
\frametitle{Conclusioni e Prospettive Future}
\begin{itemize}
    \item \textbf{Punti di forza}:
    \begin{itemize}
        \item Multimodalità nativa e finestra di contesto estesa
        \item Performance competitive su benchmark standard
        \item Integrazione nell'ecosistema Google
    \end{itemize}
    \item \textbf{Sfide}:
    \begin{itemize}
        \item Competizione intensa con GPT-4, Claude 3
        \item Bilanciamento performance vs costi computazionali
        \item Mantenimento della leadership tecnologica
    \end{itemize}
    \item \textbf{Sviluppi futuri}:
    \begin{itemize}
        \item Versioni più efficienti (Gemini Flash)
        \item Miglioramenti nella sicurezza e allineamento
        \item Espansione delle capacità multimodali
    \end{itemize}
\end{itemize}
\end{frame}
%__________________________________________________________________________
%
\section{Copilot}
%
%..................................................................
%
\begin{frame}
\frametitle{Introduzione a Microsoft Copilot}
\begin{itemize}
    \item \textbf{Sviluppatore}: Microsoft in partnership con OpenAI
    \item \textbf{Lancio}: 2021 (GitHub Copilot) $\rightarrow$ 2023 (Microsoft 365 Copilot)
    \item \textbf{Obiettivi del progetto}:
    \begin{itemize}
        \item Integrare AI generativa nell'ecosistema Microsoft
        \item Aumentare produttività in coding, office automation, business
        \item Creare un assistente AI unificato cross-platform
    \end{itemize}
    \item \textbf{Evoluzione}:
    \begin{itemize}
        \item GitHub Copilot (2021) - coding assistant
        \item Bing Chat (2023) - search + conversational AI
        \item Microsoft 365 Copilot (2023) - office productivity
        \item Windows Copilot (2023) - OS integration
    \end{itemize}
\end{itemize}
\end{frame}
%
%..................................................................
%
\begin{frame}
\frametitle{Architettura e Modelli Sottostanti}
\begin{itemize}
    \item \textbf{Modelli di base}:
    \begin{itemize}
        \item GitHub Copilot: Codex (GPT-3.5/4 specializzato per codice)
        \item Microsoft 365 Copilot: GPT-4 + modelli proprietari Microsoft
        \item Bing Copilot: GPT-4 Turbo con Prometheus (grounding)
    \end{itemize}
    \item \textbf{Architettura ibrida}:
    \begin{itemize}
        \item Retrieval-Augmented Generation (RAG)
        \item Microsoft Graph integration per dati aziendali
        \item Semantic search su contenuti Office 365
    \end{itemize}
    \item \textbf{Specializzazioni}:
    \begin{itemize}
        \item Fine-tuning per linguaggi di programmazione specifici
        \item Domain adaptation per applicazioni Office
        \item Context-aware suggestions basate su workflow
    \end{itemize}
\end{itemize}
\end{frame}
%
%..................................................................
%
\begin{frame}
\frametitle{Vantaggi e Caratteristiche Distintive}
\begin{itemize}
    \item \textbf{Integrazione profonda}:
    \begin{itemize}
        \item Nativo in Visual Studio Code, Office 365, Windows
        \item Accesso a Microsoft Graph e dati aziendali
        \item Single Sign-On e sicurezza enterprise
    \end{itemize}
    \item \textbf{Performance coding}:
    \begin{itemize}
        \item GitHub Copilot: 46\% delle righe accettate dagli sviluppatori
        \item Supporto 30+ linguaggi di programmazione
        \item Context-aware completion fino a 8KB
    \end{itemize}
    \item \textbf{Produttività office}:
    \begin{itemize}
        \item Generazione automatica PowerPoint da prompt
        \item Excel formula generation e data analysis
        \item Word drafting e summarization
        \item Outlook email composition intelligente
    \end{itemize}
\end{itemize}
\end{frame}
%
%..................................................................
%
\begin{frame}
\frametitle{Sicurezza e Governance Aziendale}
\begin{itemize}
    \item \textbf{Sicurezza dei dati}:
    \begin{itemize}
        \item Zero data retention per GitHub Copilot Business
        \item Encryption end-to-end per Microsoft 365 Copilot
        \item Compliance SOC 2 Type II, ISO 27001
    \end{itemize}
    \item \textbf{Controlli amministrativi}:
    \begin{itemize}
        \item Policy di accesso granulari per amministratori
        \item Audit logs e monitoring dell'utilizzo
        \item Data Loss Prevention (DLP) integration
    \end{itemize}
    \item \textbf{Responsabilità AI}:
    \begin{itemize}
        \item Content filtering e bias mitigation
        \item Transparency notes per ogni prodotto Copilot
        \item Responsible AI principles integration
    \end{itemize}
\end{itemize}
\end{frame}
%
%..................................................................
%
\begin{frame}
\frametitle{Confronto Dettagliato - Ecosistema}
\begin{itemize}
    \item \textbf{Microsoft Copilot vs ChatGPT}:
    \begin{itemize}
        \item Copilot: Integrazione nativa, accesso dati aziendali
        \item ChatGPT: Più generale, maggiore flessibilità conversazionale
    \end{itemize}
    \item \textbf{Microsoft Copilot vs Google Workspace AI}:
    \begin{itemize}
        \item Copilot: Più maturo, maggiore adoption enterprise
        \item Google: Integrazione Gemini in sviluppo, meno funzionalità
    \end{itemize}
    \item \textbf{GitHub Copilot vs Amazon CodeWhisperer}:
    \begin{itemize}
        \item Copilot: Maggiore market share, supporto linguaggi
        \item CodeWhisperer: Free tier, integrazione AWS, sicurezza
    \end{itemize}
    \item \textbf{Differenze chiave}: Ecosystem lock-in, enterprise focus, pricing model
\end{itemize}
\end{frame}
%
%..................................................................
%
\begin{frame}
\frametitle{Performance e Adoption Metrics}
\begin{itemize}
    \item \textbf{GitHub Copilot Statistics}:
    \begin{itemize}
        \item 1.3+ milioni di sviluppatori attivi (2024)
        \item 46\% acceptance rate per suggestions
        \item 55\% faster task completion in studi controllati
        \item 88\% developer satisfaction rate
    \end{itemize}
    \item \textbf{Microsoft 365 Copilot}:
    \begin{itemize}
        \item 70\% time saving in PowerPoint creation
        \item 50\% reduction in email composition time
        \item 60\% faster document summarization
    \end{itemize}
    \item \textbf{Enterprise adoption}:
    \begin{itemize}
        \item 1000+ aziende Fortune 500 in pilot/production
        \item ROI medio 4:1 in primo anno (studio Microsoft)
    \end{itemize}
\end{itemize}
\end{frame}
%
%..................................................................
%
\begin{frame}
\frametitle{Casi d'Uso e Applicazioni Specifiche}
\begin{itemize}
    \item \textbf{Sviluppo software}:
    \begin{itemize}
        \item Code completion e generation in IDE
        \item Unit test generation automatica
        \item Code review e security vulnerability detection
        \item Documentation generation
    \end{itemize}
    \item \textbf{Business productivity}:
    \begin{itemize}
        \item Meeting summarization in Teams
        \item Automated report generation da dati Excel
        \item Email drafting con tone adaptation
        \item Presentation creation da outline/documenti
    \end{itemize}
    \item \textbf{Data analysis}:
    \begin{itemize}
        \item Natural language to SQL queries
        \item Power BI dashboard creation
        \item Trend analysis e forecasting
    \end{itemize}
\end{itemize}
\end{frame}
%
%..................................................................
%
\begin{frame}
\frametitle{Conclusioni e Strategia Microsoft}
\begin{itemize}
    \item \textbf{Punti di forza}:
    \begin{itemize}
        \item Ecosystem integration completo e maturo
        \item Focus enterprise con security e compliance
        \item Partnership strategica con OpenAI
        \item Modello di business sostenibile
    \end{itemize}
    \item \textbf{Sfide}:
    \begin{itemize}
        \item Dipendenza da OpenAI per modelli core
        \item Pricing elevato per adoption di massa
        \item Competizione crescente da Google, Amazon
    \end{itemize}
    \item \textbf{Roadmap futura}:
    \begin{itemize}
        \item Sviluppo di modelli proprietari (MAI-1)
        \item Espansione verticali industry-specific
        \item Miglioramento multimodalità e context window
        \item Integration con mixed reality (HoloLens)
    \end{itemize}
\end{itemize}
\end{frame}
%__________________________________________________________________________
%
\section{NotebookLM}
%
%..................................................................
%
\begin{frame}
\frametitle{Che cos'è NotebookLM}
\begin{itemize}
\item NotebookLM è uno strumento di intelligenza artificiale sviluppato da Google che funziona come un assistente di ricerca personalizzato. A differenza dei chatbot tradizionali che attingono da internet, NotebookLM lavora esclusivamente sui documenti che l'utente carica, creando uno spazio di lavoro privato e specializzato.

\item La piattaforma trasforma i documenti caricati in una base di conoscenza interrogabile, permettendo di porre domande specifiche, ottenere riassunti e generare nuovi contenuti basati esclusivamente sui materiali forniti. Questo approccio garantisce che tutte le risposte siano fondate sui documenti dell'utente, eliminando il rischio di informazioni inventate o "allucinazioni" tipiche di altri sistemi AI.

\item NotebookLM rappresenta un'evoluzione nell'uso dell'intelligenza artificiale per la ricerca accademica, l'analisi aziendale e lo studio personale, offrendo un ambiente controllato e affidabile per l'elaborazione di informazioni.
\end{itemize}
\end{frame}
%
%..................................................................
%
\begin{frame}
\frametitle{A cosa serve NotebookLM}
\begin{itemize}
\item NotebookLM è progettato per supportare ricercatori, studenti, giornalisti e professionisti nell'analisi approfondita di documenti e materiali di studio. Lo strumento eccelle nell'identificare connessioni tra diversi testi, nel sintetizzare informazioni complesse e nel generare domande di approfondimento che stimolano nuove prospettive di ricerca.

\item La piattaforma si rivela particolarmente utile per preparare presentazioni, scrivere articoli di ricerca, condurre analisi comparative tra documenti e creare materiali didattici. Gli utenti possono utilizzarlo per esplorare grandi volumi di testo in modo efficiente, trovare citazioni pertinenti e sviluppare argomentazioni basate su evidenze documentali.

\item Un altro uso significativo è la preparazione di sessioni di studio o briefing professionali, dove NotebookLM può generare quiz, riassunti tematici e mappe concettuali basate sui materiali caricati, facilitando l'apprendimento e la memorizzazione delle informazioni più rilevanti.
\end{itemize}
\end{frame}
%
%..................................................................
%
\begin{frame}
\frametitle{Come utilizzare NotebookLM}
\begin{itemize}
\item L'utilizzo di NotebookLM inizia con la creazione di un "notebook" personale, uno spazio di lavoro dedicato dove caricare i documenti di interesse. La piattaforma supporta diversi formati: PDF, documenti di testo, presentazioni, pagine web e persino trascrizioni audio. Una volta caricati i materiali, l'AI analizza automaticamente i contenuti e li indicizza per facilitare le ricerche successive.

\item L'interazione avviene principalmente attraverso un'interfaccia conversazionale dove è possibile porre domande specifiche sui documenti, richiedere riassunti tematici o chiedere di evidenziare connessioni tra diversi testi. Il sistema risponde sempre citando le fonti specifiche, permettendo di verificare e approfondire ogni informazione fornita.

\item Una funzionalità distintiva è la possibilità di generare note automatiche, outline per presentazioni e persino discussioni audio simulate tra due speaker virtuali che analizzano i contenuti caricati. Questo rende lo studio più dinamico e offre prospettive diverse sui materiali esaminati.
\end{itemize}
\end{frame}
%
%..................................................................
%
\begin{frame}
\frametitle{Caratteristiche principali di NotebookLM}
\begin{itemize}
\item La caratteristica più distintiva di NotebookLM è il suo approccio "grounded" alle informazioni: ogni risposta è ancorata esclusivamente ai documenti caricati dall'utente, con citazioni precise che permettono di risalire alle fonti originali. Questo elimina il problema delle informazioni inventate, comune in altri sistemi AI.

\item Il sistema offre capacità avanzate di sintesi e analisi, riuscendo a identificare temi ricorrenti, contraddizioni e connessioni anche in collezioni di documenti molto estese. La funzione di generazione di audio overview è particolarmente innovativa: crea discussioni simulate tra due speaker che presentano e dibattono i contenuti dei documenti in formato podcast.

\item NotebookLM mantiene la privacy dei dati: i documenti caricati non vengono utilizzati per addestrare altri modelli AI e rimangono privati nell'ambiente di lavoro dell'utente. La piattaforma supporta anche la collaborazione, permettendo di condividere notebook con colleghi per lavori di gruppo, mantenendo sempre il controllo sui permessi di accesso.
\end{itemize}
\end{frame}
%
%..................................................................
%
\begin{frame}
\frametitle{Differenze con altre piattaforme AI}
\begin{itemize}
\item NotebookLM si distingue da ChatGPT e altri chatbot generici per il suo approccio specializzato e localizzato. Mentre ChatGPT attinge da una vasta base di conoscenza pre-addestrata che può contenere informazioni obsolete o incorrette, NotebookLM lavora esclusivamente sui documenti forniti dall'utente, garantendo accuratezza e pertinenza delle risposte.

\item Rispetto a strumenti come Claude o Gemini, NotebookLM offre un ambiente di lavoro strutturato e persistente. Non è progettato per conversazioni generiche, ma per l'analisi approfondita e metodica di materiali specifici. Questa specializzazione lo rende più efficace per ricerca accademica, analisi professionali e studio sistematico.

\item La differenza principale con altri strumenti di summarizzazione è la capacità di mantenere il contesto attraverso sessioni multiple e di generare contenuti multimediali come le discussioni audio. Inoltre, mentre molte piattaforme AI richiedono di riformulare domande complesse, NotebookLM comprende query sofisticate e può lavorare con insiemi di documenti correlati, offrendo un'esperienza più naturale e produttiva per lavori di ricerca intensivi.
\end{itemize}
\end{frame}
%
%..................................................................
%
\begin{frame}
\frametitle{Conclusioni e prospettive}
\begin{itemize}
\item NotebookLM rappresenta un'evoluzione significativa nell'applicazione dell'intelligenza artificiale alla ricerca e all'analisi documentale. La sua capacità di fornire risposte accurate e verificabili, basate esclusivamente sui materiali dell'utente, lo rende uno strumento prezioso per chiunque lavori con grandi quantità di informazioni testuali.

\item L'approccio privacy-first e la specializzazione nell'analisi documentale lo posizionano come complemento ideale, piuttosto che alternativa, agli altri strumenti AI disponibili. NotebookLM eccelle in contesti dove l'accuratezza, la tracciabilità delle fonti e l'analisi approfondita sono prioritarie.

\item Con il continuo sviluppo delle sue funzionalità, incluse le innovative discussioni audio e le capacità di sintesi multimediale, NotebookLM si prospetta come uno standard per il futuro della ricerca assistita dall'intelligenza artificiale, offrendo un equilibrio ottimale tra potenza tecnologica e controllo dell'utente sui propri dati e processi di lavoro.
\end{itemize}
\end{frame}
%__________________________________________________________________________
%
\section{Conclusioni}
%
%..................................................................
%
\begin{frame}
    \frametitle{Panoramica Generale sugli AI Chatbot}
    \begin{itemize}
        \item \textbf{Interfaccia Utente Standard}: Generalmente, un AI chatbot include una casella di testo per l'input dell'utente (prompt) e uno spazio per la risposta del sistema [1].
        \item \textbf{Modelli di Servizio}: La maggior parte dei servizi è offerta in modalità gratuita, con funzionalità più avanzate disponibili a pagamento [2].
        \item \textbf{Versatilità e Evoluzione}: I chatbot, come ChatGPT, sono estremamente versatili, eccellendo in contesti creativi, nella risoluzione di problemi complessi, nella scrittura, nella programmazione e ora anche nella generazione di immagini [2, 3].
        \item \textbf{Interazione Intuitiva}: L'interfaccia è essenziale e intuitiva, con risposte rapide e capacità di mantenere il contesto per conversazioni fluide e continuative [3].
    \end{itemize}
\end{frame}
%
%..................................................................
%
\begin{frame}
    \frametitle{Punti di Forza Distintivi dei Principali LLM}
    \begin{itemize}
    \small
        \item \textbf{ChatGPT}: Notoriamente \textbf{versatile} in scrittura e risoluzione problemi, utile nella programmazione e nella generazione di immagini, con un ampio ecosistema di plugin e GPT Store [2-4].
        \item \textbf{Claude}: Sviluppato con la metodologia \textbf{Constitutional AI} per sicurezza e allineamento, offre una finestra di contesto estesa (200K token) e prestazioni superiori in coding e ragionamento [4-9].
        \item \textbf{Perplexity}: Posizionato come \textbf{motore di ricerca conversazionale}, fornisce risposte dirette e verificate con fonti citate, aggiornate in tempo reale [10, 11].
        \item \textbf{Gemini}: Caratterizzato dalla \textbf{multimodalità nativa} (testo, codice, immagini, video, audio) e una finestra di contesto eccezionalmente lunga (fino a 1 milione di token), profondamente integrato nell'ecosistema Google [12-15].
        \item \textbf{Microsoft Copilot}: Integra l'AI generativa direttamente nell'ecosistema Microsoft (Office 365, Windows, GitHub), focalizzandosi sull'aumento della produttività aziendale e garantendo sicurezza e governance dei dati [16-19].
    \end{itemize}
\end{frame}
%
%..................................................................
%
\begin{frame}
    \frametitle{Prospettive Strategiche e Trend Futuri}
    \begin{itemize}
    \small
        \item \textbf{Allineamento e Sicurezza}: L'evoluzione dei modelli è sempre più guidata da metodologie di allineamento (es. Constitutional AI di Anthropic, RLHF di OpenAI e Google), con l'obiettivo di ridurre output dannosi e pregiudizievoli [5, 6, 13, 20].
        \item \textbf{Multimodalità Avanzata}: La capacità di elaborare e integrare diversi tipi di input (testo, immagini, audio, video) nativamente, come in Gemini, è fondamentale per compiti complessi e applicazioni reali [12, 21, 22].
        \item \textbf{Finestre di Contesto Estese}: La possibilità di gestire contesti conversazionali o documentali estremamente lunghi (fino a 1 milione di token per Gemini 1.5) abilita analisi approfondite e interazioni senza perdita di informazioni [8, 13].
        \item \textbf{Integrazione Ecosistemica}: La profonda integrazione in ambienti di lavoro e piattaforme esistenti (es. Microsoft 365 per Copilot, prodotti Google per Gemini) è cruciale per la mass adoption e l'efficienza professionale [14, 17, 19].
        \item \textbf{Co-ottimizzazione Performance e Sicurezza}: Il futuro dei Large Language Models dimostra che performance elevate e sicurezza non sono obiettivi mutuamente esclusivi, ma possono essere ottimizzati congiuntamente attraverso approcci basati su controlli [23].
    \end{itemize}
\end{frame}
%
%=====================================================================
%
\end{document}
%
%=====================================================================
%
